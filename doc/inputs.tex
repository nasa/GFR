
%\documentclass[]{aiaa-tc}% insert '[draft]' option to show overfull boxes
%\documentclass[10pt,letterpaper]{article}% 
%                 insert '[draft]' option to show overfull boxes

\documentclass[letterpaper,10pt]{article}
\usepackage[paper=letterpaper,landscape,centering]{geometry}
%\usepackage[paper=letterpaper,centering]{geometry}

\usepackage{amsmath}
%\usepackage{amsfonts}
%\usepackage[mathscr]{eucal}
\usepackage{longtable}
\usepackage{tabularx}
% \usepackage{threeparttable}% tables with footnotes
\usepackage{booktabs}
 \usepackage[x11names]{xcolor}
 \usepackage{graphics}
%\usepackage{graphicx}
% %
% \usepackage{listings}
% \usepackage{multirow}
%%\usepackage{array}
%%\usepackage{xifthen}
% \usepackage{mathtools}
% \usepackage{siunitx}
\usepackage{printlen}
\usepackage{calc}
\usepackage{url}
%

\DeclareMathAlphabet{\mathsc}{OT1}{cmr}{m}{sc}

%%\setlength{\paperwidth}{614.295pt}
%%\setlength{\paperheight}{794.96999pt}
%\addtolength{\textwidth}{0.5in}
%%\setlength{\textheight}{556.47656pt}
\setlength{\hoffset}{-0.5in}
\setlength{\voffset}{-0.5in}

\setlength{\textwidth}{\paperwidth-2in-2.0\hoffset}
%\setlength{\textwidth}{\paperwidth}
%\addtolength{\textwidth}{\hoffset}
%\addtolength{\textwidth}{\hoffset}

\setlength{\textheight}{\paperheight-2in-2.0\voffset}
%\setlength{\textheight}{\paperheight}
%\addtolength{\textheight}{\voffset}
%\addtolength{\textheight}{\voffset}

\setlength{\oddsidemargin}{0pt}
\setlength{\evensidemargin}{0pt}
%\setlength{\topmargin}{9.97672pt}
\setlength{\topmargin}{0pt}
\setlength{\headheight}{0pt}
\setlength{\headsep}{0pt}
\setlength{\topskip}{0pt}
\setlength{\footskip}{20pt}
\setlength{\marginparwidth}{0pt}
\setlength{\marginparsep}{0pt}
\setlength{\marginparpush}{0pt}
\setlength{\columnsep}{0pt}

%
 \newcommand{\slbsc}{basic}
 \newcommand{\sladv}{advanced}
 \newcommand{\slxpt}{expert}
 \newcommand{\slxtl}{experimental}
 \newcommand{\sldev}{developer}
 \newcommand{\slnwk}{not working}
 \newcommand{\slspc}{special case}
 \newcommand{\slutl}{utility}
%
 \newcommand{\typint}{integer}
 \newcommand{\typflt}{integer}
 \newcommand{\typlog}{integer}
 \newcommand{\tc}[1][150]{character(len=#1)}
 \newcommand{\typebc}{bc\_input\_t}
%
 \newcommand{\tru}{.TRUE.}
 \newcommand{\fls}{.FALSE.}
%
 \newcommand{\minorline}{\hline}
%\newcommand{\groupline}[1]{\multicolumn{5}{|c|}{#1}\\\hline}
 \newcommand{\groupline}[1]{}
% \newcommand{\p}[1]{\ensuremath{\mathcal{P}#1}}
% \newcommand{\dof}{DoF}
% \newcommand{\tstar}{\ensuremath{t^{*}}}
% \newcommand{\kedrz}{\ensuremath{\epsilon(\zeta)}}

\newcommand{\headersize}[1]{\large\textbf{#1}}

\newlength{\colAwidth}
\newlength{\colBwidth}
\newlength{\colCwidth}
\newlength{\colDwidth}
\newlength{\colEwidth}

\setlength{\colAwidth}{\maxof{\widthof{\headersize{Input Variable}}}
                             {\widthof{interpolate\_before\_output\_variable}}}
\setlength{\colBwidth}{\maxof{\widthof{\headersize{Type}}}
                             {\widthof{\tc}}}
\setlength{\colCwidth}{\maxof{\widthof{\headersize{Default Value}}}
                             {\widthof{\fls}}}
\setlength{\colDwidth}{\maxof{\widthof{\headersize{Skill Level}}}
                             {\widthof{experimental}}}
\setlength{\colEwidth}
          {\textwidth-\colAwidth-\colBwidth-\colCwidth-\colDwidth-62.4pt}
%\setlength{\colEwidth}{\textwidth-\colAwidth-\colBwidth-\colCwidth-\colDwidth}

 \newcommand{\descriptionbegin}{}
 \newcommand{\descriptionend}{\\ \minorline}

 \newcommand{\thatis}{i.e.,~}
 \newcommand{\forexample}{e.g.,~}

%\definecolor{orange}{
 \newcommand{\p}[2][]{\ensuremath{\mathcal{P}_{\textsl{#2}}#1}}
 \newcommand{\WARNING}{\textcolor{red}{\textbf{WARNING: }}}
 \newcommand{\NOTE}{\newline \textcolor{OrangeRed3}{\textbf{NOTE: }}}
 \newcommand{\NoNewlineNOTE}{\textcolor{OrangeRed3}{\textbf{NOTE: }}}
%\newcommand{\NOTE}{\newline\fcolorbox{white}{yellow}{NOTE:}}
 \newcommand{\EXAMPLE}{\newline \textcolor{blue}{\textbf{EXAMPLE: }}}

 \newcommand{\eql}{\ensuremath{=\!\!\;=}}

%\newcommand{\descriptionbegin}
%           {\begin{minipage}[t]{\linewidth}\begin{flushleft}}
%\newcommand{\descriptionend}{\end{flushleft}\end{minipage} \\ \minorline}

\begin{document}

 \setlength{\abovedisplayskip}{4pt}
 \setlength{\belowdisplayskip}{4pt}

%\begin{center}
   %\begin{tabular}{ | p{0.2\textwidth}
   %\begin{longtable}{ | p{0.265\textwidth}
   %                   | p{0.145\textwidth}
   %                   | p{0.1\textwidth}
   %\begin{longtable}{ | l | c | c | c | p{0.45\textwidth} | }
    \begin{longtable}{ | l | c | c | c | p{\colEwidth} | }
   %\begin{longtable}{ | l | c | c | X | }
   %\begin{tabularx}{\textwidth}{ | l | c | c | X | }
    \hline
    \rule{0pt}{5mm}
    \headersize{Input Variable} &
    \headersize{Type} &
    \headersize{Default Value} &
    \headersize{Skill Level} &
    \headersize{Description} \\ \hline

    %%%%%%%%%%%%%%%%%%%%%%%%%%%%%%%%%%%%%%%%%%%%%%%%%%%%%%%%%%%%%%%%%%%%%%%%%%%%
    %%%%%%%%%%%%%%%%%%%%%%%%%%%%%%%%%%%%%%%%%%%%%%%%%%%%%%%%%%%%%%%%%%%%%%%%%%%%
    %%%%%%%%%%%%%%%%%%%%%%%%%%%%%%%%%%%%%%%%%%%%%%%%%%%%%%%%%%%%%%%%%%%%%%%%%%%%
    \groupline{GRID INFORMATION}
    %===========================================================================
    grid\_format          & \typint & -1    & \slbsc &
    \begin{minipage}[t]{\linewidth}\begin{flushleft}
    Specifies the format of the input grid file.
    \begin{tabular}{ @{\qquad} r @{ = } p{0.85\linewidth} @{} }
    -1 & Internally generated grid (automatically set for certain test cases) \\
    1 & Gambit Neutral \\
    2 & Gmsh \\
    3 & Plot3D \\
    4 & CGNS
    \end{tabular}
    \end{flushleft}\end{minipage} \\ \minorline
    %---------------------------------------------------------------------------
    gridfile              & \tc     & `` '' & \slbsc &
    \descriptionbegin
    File path to the input grid file relative to the current directory.
    \descriptionend
    %---------------------------------------------------------------------------
    plot3d\_cutfile       & \tc     & `` '' & \slbsc &
    \descriptionbegin
    File path to the input file that specifies the cut conditions for a
    multiblock Plot3D grid file.
    \descriptionend
    %---------------------------------------------------------------------------
    plot3d\_bcsfile       & \tc     & `` '' & \slbsc &
    \descriptionbegin
    File path to the input file that specifies the boundary conditions for a
    Plot3D grid file.
    \descriptionend
    %---------------------------------------------------------------------------
    grid\_scaling\_factor & \typflt & 1.0   & \slbsc &
    \descriptionbegin
    Multiplication factor ($> 0.0$) to scale the grid coordinates to the
    desired length units. For example, use a value of 0.0254 to convert a grid
    containing coordinates in inches to meters.
    \descriptionend
    %%%%%%%%%%%%%%%%%%%%%%%%%%%%%%%%%%%%%%%%%%%%%%%%%%%%%%%%%%%%%%%%%%%%%%%%%%%%

    %%%%%%%%%%%%%%%%%%%%%%%%%%%%%%%%%%%%%%%%%%%%%%%%%%%%%%%%%%%%%%%%%%%%%%%%%%%%
    %%%%%%%%%%%%%%%%%%%%%%%%%%%%%%%%%%%%%%%%%%%%%%%%%%%%%%%%%%%%%%%%%%%%%%%%%%%%
    %%%%%%%%%%%%%%%%%%%%%%%%%%%%%%%%%%%%%%%%%%%%%%%%%%%%%%%%%%%%%%%%%%%%%%%%%%%%
    \groupline{PLOT3D COARSENING OPTIONS}
    %===========================================================================
    plot3d\_i\_stride          & \typint & -1 & \sladv &
    \descriptionbegin
    If $> 0$, applies a coarsening stride to the \textit{\textbf{i}}-direction
    of all Plot3D blocks. 
    \NOTE This overrides the \textit{\textbf{i}}-direction coarsening
    stride given by \textsl{plot3d\_all\_strides}.
    \descriptionend
    %---------------------------------------------------------------------------
    plot3d\_j\_stride          & \typint & -1 & \sladv &
    \descriptionbegin
    If $> 0$, applies a coarsening stride to the \textit{\textbf{j}}-direction
    of all Plot3D blocks. 
    \NOTE This overrides the \textit{\textbf{j}}-direction coarsening
    stride given by \textsl{plot3d\_all\_strides}.
    \descriptionend
    %---------------------------------------------------------------------------
    plot3d\_k\_stride          & \typint & -1 & \sladv &
    \descriptionbegin
    If $> 0$, applies a coarsening stride to the \textit{\textbf{k}}-direction
    of all Plot3D blocks. 
    \NOTE This overrides the \textit{\textbf{k}}-direction coarsening
    stride given by \textsl{plot3d\_all\_strides}.
    \descriptionend
    %---------------------------------------------------------------------------
    plot3d\_all\_strides       & \typint & -1 & \sladv &
    \descriptionbegin
    Applies a coarsening stride to all \textit{\textbf{i}}, \textit{\textbf{j}},
    \textit{\textbf{k}} directions of all Plot3D blocks. 
    \NOTE The coarsening stride for any given direction can be
    overridden by specifying the coarsening stride for that direction.
    \descriptionend
    %---------------------------------------------------------------------------
    plot3d\_agglomerate\_order & \typint & -1 & \slbsc &
    \descriptionbegin
    If $> 0$, agglomerates linear Plot3D cells into high-order grid cells of
    this order. 
    \NOTE For all blocks, the number of cells in each direction must be
    divisible by this order.
    \descriptionend
    %%%%%%%%%%%%%%%%%%%%%%%%%%%%%%%%%%%%%%%%%%%%%%%%%%%%%%%%%%%%%%%%%%%%%%%%%%%%

    %%%%%%%%%%%%%%%%%%%%%%%%%%%%%%%%%%%%%%%%%%%%%%%%%%%%%%%%%%%%%%%%%%%%%%%%%%%%
    %%%%%%%%%%%%%%%%%%%%%%%%%%%%%%%%%%%%%%%%%%%%%%%%%%%%%%%%%%%%%%%%%%%%%%%%%%%%
    %%%%%%%%%%%%%%%%%%%%%%%%%%%%%%%%%%%%%%%%%%%%%%%%%%%%%%%%%%%%%%%%%%%%%%%%%%%%
    \groupline{BASE FR OPTIONS}
    %===========================================================================
    solution\_order      & \typint & 3 & \slbsc &
    \descriptionbegin
    Specifies the degree of the polynomial space, \p{S}, to use for the 
    solution. 
    \NOTE The order-of-accuracy is approximately \p[+1]{S}.
    \descriptionend
    %---------------------------------------------------------------------------
    flux\_divergence     & \typint & 1 & \slnwk &
    \begin{minipage}[t]{\linewidth}\begin{flushleft}
    Specifies the method to use for computing the divergence of the fluxes. \\
    \begin{tabular}{ @{} r @{ = } p{0.85\linewidth} @{} }
    \ensuremath{\qquad1} & using Lagrange polynomials \\
    \ensuremath{\qquad2} & using chain rule \\
    \ensuremath{\qquad3} & using chain rule with Jacobians
    \end{tabular}
    \end{flushleft}\end{minipage} \\ \minorline
    %---------------------------------------------------------------------------
    correction\_function & \typint & 1 & \sladv &
    \begin{minipage}[t]{\linewidth}\begin{flushleft}
    Specifies the correction function to use.
    \begin{tabular}{ @{\qquad} r @{ = } p{0.85\linewidth} @{} }
    1 & \textsl{g}\textsubscript{DG} correction function (recovers nodal DG
    method) \\
    2 & \textsl{g}\textsubscript{2} correction function \\
    3 & \textsl{g}\textsubscript{Ga} correction function
    \end{tabular}
    \end{flushleft}\end{minipage} \\ \minorline
    %%%%%%%%%%%%%%%%%%%%%%%%%%%%%%%%%%%%%%%%%%%%%%%%%%%%%%%%%%%%%%%%%%%%%%%%%%%%

    %%%%%%%%%%%%%%%%%%%%%%%%%%%%%%%%%%%%%%%%%%%%%%%%%%%%%%%%%%%%%%%%%%%%%%%%%%%%
    %%%%%%%%%%%%%%%%%%%%%%%%%%%%%%%%%%%%%%%%%%%%%%%%%%%%%%%%%%%%%%%%%%%%%%%%%%%%
    %%%%%%%%%%%%%%%%%%%%%%%%%%%%%%%%%%%%%%%%%%%%%%%%%%%%%%%%%%%%%%%%%%%%%%%%%%%%
    \groupline{LOCATION OF SOLUTION, FLUX, AND TRIANGULAR POINTS}
    %===========================================================================
    loc\_solution\_pts & \typint & 1 & \sladv &
    \begin{minipage}[t]{\linewidth}\begin{flushleft}
    Specifies the location of the solution points used for tensor products.
    \begin{tabular}{ @{\qquad} r @{ = } p{0.85\linewidth} @{} }
    1 & Legendre-Gauss nodes \\
    2 & Legendre-Gauss-Lobatto nodes
    \end{tabular}
    \end{flushleft}\end{minipage} \\ \minorline
    %---------------------------------------------------------------------------
    loc\_flux\_pts     & \typint & 1 & \sladv &
    \begin{minipage}[t]{\linewidth}\begin{flushleft}
    Specifies the location of the flux points on cell faces.
    \begin{tabular}{ @{\qquad} r @{ = } p{0.85\linewidth} @{} }
    1 & Legendre-Gauss nodes \\
    2 & Legendre-Gauss-Lobatto nodes
    \end{tabular}
    \end{flushleft}\end{minipage} \\ \minorline
    %---------------------------------------------------------------------------
    loc\_triangle\_pts & \typint & 2 & \sldev &
    \begin{minipage}[t]{\linewidth}\begin{flushleft}
    Specifies the location of the solution points used for triangles.
    \begin{tabular}{ @{\qquad} r @{ = } p{0.85\linewidth} @{} }
    1 & Hesthaven-Warburton $\alpha$-optimized nodes \\
    2 & Barycentric Lobatto nodes
    \end{tabular}
    \end{flushleft}\end{minipage} \\ \minorline
    %%%%%%%%%%%%%%%%%%%%%%%%%%%%%%%%%%%%%%%%%%%%%%%%%%%%%%%%%%%%%%%%%%%%%%%%%%%%

    %%%%%%%%%%%%%%%%%%%%%%%%%%%%%%%%%%%%%%%%%%%%%%%%%%%%%%%%%%%%%%%%%%%%%%%%%%%%
    %%%%%%%%%%%%%%%%%%%%%%%%%%%%%%%%%%%%%%%%%%%%%%%%%%%%%%%%%%%%%%%%%%%%%%%%%%%%
    %%%%%%%%%%%%%%%%%%%%%%%%%%%%%%%%%%%%%%%%%%%%%%%%%%%%%%%%%%%%%%%%%%%%%%%%%%%%
    \groupline{LOCATION OF INITIALIZATION POINTS}
    %===========================================================================
    loc\_init\_pts & \typint & 0 & \slxpt &
    \begin{minipage}[t]{\linewidth}\begin{flushleft}
    Nodal points used for initializing the flow, after which the initial
    solution is projected to the solution points.
    \begin{tabular}{ @{\qquad} r @{ = } p{0.85\linewidth} @{} }
    1 & Legendre-Gauss nodes \\
    2 & Legendre-Gauss-Lobatto nodes
    \end{tabular}
    \NOTE Any other values besides these will result in
    $\textsl{loc\_init\_pts}=\textsl{loc\_solution\_pts}$.
    \end{flushleft}\end{minipage} \\ \minorline
    %%%%%%%%%%%%%%%%%%%%%%%%%%%%%%%%%%%%%%%%%%%%%%%%%%%%%%%%%%%%%%%%%%%%%%%%%%%%

    %%%%%%%%%%%%%%%%%%%%%%%%%%%%%%%%%%%%%%%%%%%%%%%%%%%%%%%%%%%%%%%%%%%%%%%%%%%%
    %%%%%%%%%%%%%%%%%%%%%%%%%%%%%%%%%%%%%%%%%%%%%%%%%%%%%%%%%%%%%%%%%%%%%%%%%%%%
    %%%%%%%%%%%%%%%%%%%%%%%%%%%%%%%%%%%%%%%%%%%%%%%%%%%%%%%%%%%%%%%%%%%%%%%%%%%%
    \groupline{RUNGE-KUTTA OPTIONS}
    %===========================================================================
    Runge\_Kutta\_Scheme & \typint & 3 & \slbsc &
    \begin{minipage}[t]{\linewidth}\begin{flushleft}
    Specifies which Runge-Kutta method to use.
    \begin{tabular}{ @{\qquad} r @{ = } p{0.85\linewidth} @{} }
    1 & Classic $n$-stage RK method \\
    2 & 2-stage/2\textsuperscript{nd}-order SSP-RK method \\
    3 & 3-stage/3\textsuperscript{rd}-order SSP-RK method \\
    4 & 5-stage/4\textsuperscript{th}-order SSP-RK method \\
    5 & 5-stage/4\textsuperscript{th}-order Carpenter-Kennedy low storage RK
    method
    \end{tabular}
    \end{flushleft}\end{minipage} \\ \minorline
    %---------------------------------------------------------------------------
    num\_rk\_stages      & \typint & 3 & \slbsc &
    \descriptionbegin
    Gives the number of Runge-Kutta stages if using the classic RK method
    $\left(\textsl{Runge\_Kutta\_Scheme} = 1\right)$.
    \descriptionend
    %%%%%%%%%%%%%%%%%%%%%%%%%%%%%%%%%%%%%%%%%%%%%%%%%%%%%%%%%%%%%%%%%%%%%%%%%%%%

    %%%%%%%%%%%%%%%%%%%%%%%%%%%%%%%%%%%%%%%%%%%%%%%%%%%%%%%%%%%%%%%%%%%%%%%%%%%%
    %%%%%%%%%%%%%%%%%%%%%%%%%%%%%%%%%%%%%%%%%%%%%%%%%%%%%%%%%%%%%%%%%%%%%%%%%%%%
    %%%%%%%%%%%%%%%%%%%%%%%%%%%%%%%%%%%%%%%%%%%%%%%%%%%%%%%%%%%%%%%%%%%%%%%%%%%%
    \groupline{COMMON FLUX OPTIONS}
    %===========================================================================
    invs\_flux\_method       & \typint & 1 & \sladv &
    \begin{minipage}[t]{\linewidth}\begin{flushleft}
    Specifies the approximate Riemann solver used to compute the common inviscid
    fluxes at the interfaces.
    \begin{tabular}{ @{\qquad} r @{ = } p{0.85\linewidth} @{} }
    1 & Roe flux with entropy fix \\
    2 & HLLC flux \\
    3 & LDFSS flux \\
    4 & Rotated-Roe-HLL flux
    \end{tabular}
    \end{flushleft}\end{minipage} \\ \minorline
    %---------------------------------------------------------------------------
    visc\_flux\_method       & \typint & 2 & \sladv &
    \begin{minipage}[t]{\linewidth}\begin{flushleft}
    Specifies the method used to compute the common viscous fluxes at the
    interfaces.
    \begin{tabular}{ @{\qquad} r @{ = } p{0.85\linewidth} @{} }
    1 & Bassi-Rebay 1 (BR1) (\WARNING high chances of stability issues) \\
    2 & Bassi-Rebay 2 (BR2) \\
    3 & Local DG (LDG) (smaller CFL limit than BR2)
    \end{tabular}
    \end{flushleft}\end{minipage} \\ \minorline
    %---------------------------------------------------------------------------
    visc\_variables\_to\_ave & \typint & 1 & \slxpt &
    \begin{minipage}[t]{\linewidth}\begin{flushleft}
    Determines which variables are averaged to the interfaces when computing the
    common viscous fluxes.
    \begin{tabular}{ @{\qquad} r @{ = } p{0.85\linewidth} @{} }
    1 & Average the interface fluxes. Compute the common viscous flux by
    averaging the left and right viscous interface fluxes which are computed
    using the left and right interface solutions and gradients. \\
    2 & Average the interface solutions and gradients. Compute common solutions
    and gradients by averaging the left and right interface solutions and
    gradients, and then use the common interface values to compute the common
    viscous flux.
    \end{tabular}
    \end{flushleft}\end{minipage} \\ \minorline
    %%%%%%%%%%%%%%%%%%%%%%%%%%%%%%%%%%%%%%%%%%%%%%%%%%%%%%%%%%%%%%%%%%%%%%%%%%%%

    %%%%%%%%%%%%%%%%%%%%%%%%%%%%%%%%%%%%%%%%%%%%%%%%%%%%%%%%%%%%%%%%%%%%%%%%%%%%
    %%%%%%%%%%%%%%%%%%%%%%%%%%%%%%%%%%%%%%%%%%%%%%%%%%%%%%%%%%%%%%%%%%%%%%%%%%%%
    %%%%%%%%%%%%%%%%%%%%%%%%%%%%%%%%%%%%%%%%%%%%%%%%%%%%%%%%%%%%%%%%%%%%%%%%%%%%
    \groupline{TIME STEPPING INFORMATION}
    %===========================================================================
    num\_timesteps     & \typint & 25000  & \slbsc &
    \descriptionbegin
    Number of time steps for the simulation.
    \descriptionend
    %---------------------------------------------------------------------------
    Final\_Time        & \typflt & 10.0   & \slbsc &
    \descriptionbegin
    Stop when the time within the simulation reaches this value.
    \descriptionend
    %---------------------------------------------------------------------------
    constant\_dt       & \typflt & 1.0e-6 & \slbsc &
    \descriptionbegin
    Size of the time step that is used when \textsl{Timestep\_Type} is set to a
    constant global time step.
    \descriptionend
    %---------------------------------------------------------------------------
    Timestep\_Type     & \typint & 1      & \slbsc &
    \begin{minipage}[t]{\linewidth}\begin{flushleft}
    Type of time stepping to use for the simulation.
    \begin{tabular}{ @{\qquad} r @{ = } p{0.85\linewidth} @{} }
    0 & Constant global time step. \\
    1 & global time step from minimum computed cell time step. \\
    -1 & local time stepping within each cell from computed cell time step. \\
    2 & global time step from minimum computed solution point time step.
    (EXPERIMENTAL) \\
    -2 & local time stepping at each solution point from compute solution point
    time step. (EXPERIMENTAL)
    \end{tabular}
    \end{flushleft}\end{minipage} \\ \minorline
    %---------------------------------------------------------------------------
    minimum\_timesteps & \typint & 100    & \slbsc &
    \descriptionbegin
    The minimum number of time steps to complete before checking whether any of
    the convergence criteria have been met. 
    \NOTE When initializing a simulation to a uniform flow, the very
    small changes that occur during the first few time steps can trigger a false
    positive convergence check if this is set to a small number, \forexample
    $\leq5$.
    \descriptionend
    %%%%%%%%%%%%%%%%%%%%%%%%%%%%%%%%%%%%%%%%%%%%%%%%%%%%%%%%%%%%%%%%%%%%%%%%%%%%

    %%%%%%%%%%%%%%%%%%%%%%%%%%%%%%%%%%%%%%%%%%%%%%%%%%%%%%%%%%%%%%%%%%%%%%%%%%%%
    %%%%%%%%%%%%%%%%%%%%%%%%%%%%%%%%%%%%%%%%%%%%%%%%%%%%%%%%%%%%%%%%%%%%%%%%%%%%
    %%%%%%%%%%%%%%%%%%%%%%%%%%%%%%%%%%%%%%%%%%%%%%%%%%%%%%%%%%%%%%%%%%%%%%%%%%%%
    \groupline{CFL CONTROL VARIABLES}
    %===========================================================================
    CFL         & \typflt & 0.1 & \slbsc &
    \descriptionbegin
    Analogous to the CFL condition, this is used to scale the time step size to
    get a maximum stable time step. The maximum value this can be while also
    keeping a simulation stable is very dependent on the simulation\slash
    flow-conditions and input settings, and experimentation is generally
    required to find a good value. A lower value is more likely to keep a
    simulation stable, but it will require more time steps to reach either
    convergence or a given simulation time. Typical values are
    $0.1\leq\textsl{CFL}\leq2$, but past experience has found simulations that
    were stable with values up to 7.
    \descriptionend
    %---------------------------------------------------------------------------
    CFL\_Beg    & \typflt & 0.1 & \slbsc &
    \descriptionbegin
    Beginning \textsl{CFL} value to use with the linear CFL ramping that is
    active when $\textsl{CFL\_Cycles}>2$.
    \descriptionend
    %---------------------------------------------------------------------------
    CFL\_End    & \typflt & 0.1 & \slbsc &
    \descriptionbegin
    Ending \textsl{CFL} value to use with the linear CFL ramping that is active
    when $\textsl{CFL\_Cycles}>2$.
    \descriptionend
    %---------------------------------------------------------------------------
    CFL\_Cycles & \typint & 1   & \slbsc &
    \descriptionbegin
    The number of time steps used to linearly ramp up the value of \textsl{CFL},
    starting at \textsl{CFL\_Beg} and ending at \textsl{CFL\_End}.
    \descriptionend
    %%%%%%%%%%%%%%%%%%%%%%%%%%%%%%%%%%%%%%%%%%%%%%%%%%%%%%%%%%%%%%%%%%%%%%%%%%%%

    %%%%%%%%%%%%%%%%%%%%%%%%%%%%%%%%%%%%%%%%%%%%%%%%%%%%%%%%%%%%%%%%%%%%%%%%%%%%
    %%%%%%%%%%%%%%%%%%%%%%%%%%%%%%%%%%%%%%%%%%%%%%%%%%%%%%%%%%%%%%%%%%%%%%%%%%%%
    %%%%%%%%%%%%%%%%%%%%%%%%%%%%%%%%%%%%%%%%%%%%%%%%%%%%%%%%%%%%%%%%%%%%%%%%%%%%
    \groupline{OVER-INTEGRATION OPTIONS}
    %===========================================================================
    projection\_order    & \typint & 3 & \sladv &
    \descriptionbegin
    Specifies the degree of the polynomial space, \p{P}, to project the residual
    onto when using over-integration by residual projection. The residual will
    be computed as a polynomial of degree \p{S} and projected down to a
    polynomial of degree \p{P}. This can potentially improve stability if
    aliasing errors are suspected of destabilizing a simulation. 
    \NOTE This is only used when
    $\textsl{projection\_order}<\textsl{solution\_order}$. 
    \NOTE Using this means the order-of-accuracy of the simulation is
    now approximately \p[+1]{P} instead of \p[+1]{S}.
    \descriptionend
    %---------------------------------------------------------------------------
    quadrature\_order    & \typint & 3 & \slnwk &
    \descriptionbegin
    Specifies the degree of the quadrature polynomial space, \p{Q}, to use when
    using over-integration by over-sampling at the quadrature points. Certain
    operations during a given residual evaluation (\forexample evaluating the
    common\slash upwind interface fluxes) produce either polynomials that are
    not within the solution polynomial space or functions that are not
    polynomials at all. These functions are over-sampled at the quadrature
    points to produce approximations of these functions as polynomials of degree
    \p{Q}, which are in turn projected down to polynomials of degree \p{S}
    before adding their respective contribution to the residual. This can
    potentially improve stability if aliasing errors are suspected of
    destabilizing a simulation.
    \NOTE $\textsl{quadrature\_order}\geq\textsl{solution\_order}$
    \NOTE This is only used when
    \[\textrm{or}\qquad \begin{array}{l}
        \textsl{quadrature\_order}>\textsl{solution\_order} \\
        \textsl{loc\_quadrature\_pts}\neq\textsl{loc\_solution\_pts}
    \end{array}\]
    \NoNewlineNOTE Unlike over-integration by residual projection, the
    order-of-accuracy of the simulation remains \p[+1]{S}.
    \descriptionend
    %---------------------------------------------------------------------------
    loc\_quadrature\_pts & \typint & 1 & \slnwk &
    \begin{minipage}[t]{\linewidth}\begin{flushleft}
    Specifies the location of the quadrature points used for over-integration by
    over-sampling at the quadrature points.
    \begin{tabular}{ @{\qquad} r @{ = } p{0.85\linewidth} @{} }
    1 & Legendre-Gauss nodes \\
    2 & Legendre-Gauss-Lobatto nodes
    \end{tabular}
    \end{flushleft}\end{minipage} \\ \minorline
    %%%%%%%%%%%%%%%%%%%%%%%%%%%%%%%%%%%%%%%%%%%%%%%%%%%%%%%%%%%%%%%%%%%%%%%%%%%%

    %%%%%%%%%%%%%%%%%%%%%%%%%%%%%%%%%%%%%%%%%%%%%%%%%%%%%%%%%%%%%%%%%%%%%%%%%%%%
    %%%%%%%%%%%%%%%%%%%%%%%%%%%%%%%%%%%%%%%%%%%%%%%%%%%%%%%%%%%%%%%%%%%%%%%%%%%%
    %%%%%%%%%%%%%%%%%%%%%%%%%%%%%%%%%%%%%%%%%%%%%%%%%%%%%%%%%%%%%%%%%%%%%%%%%%%%
    \groupline{ERROR-INTEGRATION OPTIONS}
    %===========================================================================
    error\_order    & \typint & -3 & \sladv &
    \descriptionbegin
    Specifies the degree of the polynomial space to use when computing
    integrated error quantities, \forexample the $L^2$-norm of the residual.
    \NOTE $\textsl{error\_order}\geq\textsl{solution\_order}$
    \NOTE This is only used when
    \[\textrm{or}\qquad \begin{array}{l}
        \textsl{error\_order}>\textsl{solution\_order} \\
        \textsl{loc\_error\_pts}\neq\textsl{loc\_solution\_pts}
    \end{array}\]
    \descriptionend
    %---------------------------------------------------------------------------
    loc\_error\_pts & \typint & 1  & \sladv &
    \begin{minipage}[t]{\linewidth}\begin{flushleft}
    Specifies the location of the error points used for computing integrated
    error quantities. 
    \begin{tabular}{ @{\qquad} r @{ = } p{0.85\linewidth} @{} }
    1 & Legendre-Gauss nodes \\
    2 & Legendre-Gauss-Lobatto nodes
    \end{tabular}
    \NOTE Any other values besides these will result in
    $\textsl{loc\_error\_pts}=\textsl{loc\_solution\_pts}$.
    \end{flushleft}\end{minipage} \\ \minorline
    %%%%%%%%%%%%%%%%%%%%%%%%%%%%%%%%%%%%%%%%%%%%%%%%%%%%%%%%%%%%%%%%%%%%%%%%%%%%

    %%%%%%%%%%%%%%%%%%%%%%%%%%%%%%%%%%%%%%%%%%%%%%%%%%%%%%%%%%%%%%%%%%%%%%%%%%%%
    %%%%%%%%%%%%%%%%%%%%%%%%%%%%%%%%%%%%%%%%%%%%%%%%%%%%%%%%%%%%%%%%%%%%%%%%%%%%
    %%%%%%%%%%%%%%%%%%%%%%%%%%%%%%%%%%%%%%%%%%%%%%%%%%%%%%%%%%%%%%%%%%%%%%%%%%%%
    \groupline{FORCE GLOBAL ORDER AND SOLUTION POINTS}
    %===========================================================================
    all\_orders & \typint & -3 & \slbsc &
    \descriptionbegin
    If $> 0$, sets \textsl{solution\_order}, \textsl{projection\_order}, and
    \textsl{quadrature\_order} all to this value.
    \NOTE This overrides the values given for these three input
    parameters.
    \descriptionend
    %---------------------------------------------------------------------------
    all\_points & \typint & -1 & \sladv &
    \begin{minipage}[t]{\linewidth}\begin{flushleft}
    Sets \textsl{loc\_solution\_pts}, \textsl{loc\_flux\_pts}, and
    \textsl{loc\_quadrature\_pts} all to this value.
    \begin{tabular}{ @{\qquad} r @{ = } p{0.85\linewidth} @{} }
    1 & Legendre-Gauss nodes \\
    2 & Legendre-Gauss-Lobatto nodes
    \end{tabular}
    \NOTE This overrides the values given for the three mentioned input
    parameters. It is only used if equal to one of the above values; otherwise,
    the three mentioned input parameters are unchanged.
    \end{flushleft}\end{minipage} \\ \minorline
    %%%%%%%%%%%%%%%%%%%%%%%%%%%%%%%%%%%%%%%%%%%%%%%%%%%%%%%%%%%%%%%%%%%%%%%%%%%%

    %%%%%%%%%%%%%%%%%%%%%%%%%%%%%%%%%%%%%%%%%%%%%%%%%%%%%%%%%%%%%%%%%%%%%%%%%%%%
    %%%%%%%%%%%%%%%%%%%%%%%%%%%%%%%%%%%%%%%%%%%%%%%%%%%%%%%%%%%%%%%%%%%%%%%%%%%%
    %%%%%%%%%%%%%%%%%%%%%%%%%%%%%%%%%%%%%%%%%%%%%%%%%%%%%%%%%%%%%%%%%%%%%%%%%%%%
    \groupline{PARALLEL OPTIONS}
    %===========================================================================
    metis\_option & \typint & 3 & \slxpt &
    \begin{minipage}[t]{\linewidth}\begin{flushleft}
    Specifies the METIS partitioning algorithm to use for partitioning the
    global grid.
    \begin{tabular}{ @{\qquad} r @{ = } p{0.85\linewidth} @{} }
    1 & Create non-weighted partitions based on a multilevel recursive bisection
    while minimizing the edge cut. \\
    2 & Create non-weighted partitions based on a multilevel $k$-way algorithm
    while minimizing the edge cut. \\
    3 & Create non-weighted partitions based on a multilevel $k$-way algorithm
    while minimizing the total communication volume. \\
    4 & Create weighted partitions based on a multilevel $k$-way algorithm
    while minimizing the total communication volume (weighting is not yet
    implemented so this is currently the same as option 3).
    \end{tabular}
    \end{flushleft}\end{minipage} \\ \minorline
    %%%%%%%%%%%%%%%%%%%%%%%%%%%%%%%%%%%%%%%%%%%%%%%%%%%%%%%%%%%%%%%%%%%%%%%%%%%%

    %%%%%%%%%%%%%%%%%%%%%%%%%%%%%%%%%%%%%%%%%%%%%%%%%%%%%%%%%%%%%%%%%%%%%%%%%%%%
    %%%%%%%%%%%%%%%%%%%%%%%%%%%%%%%%%%%%%%%%%%%%%%%%%%%%%%%%%%%%%%%%%%%%%%%%%%%%
    %%%%%%%%%%%%%%%%%%%%%%%%%%%%%%%%%%%%%%%%%%%%%%%%%%%%%%%%%%%%%%%%%%%%%%%%%%%%
    \groupline{METHOD OF MANUFACTURED SOLUTIONS (MMS) OPTIONS}
    %===========================================================================
    mms\_opt              & \typint & 0    & \slxtl &
    \descriptionbegin
    Specifies the method of manufactured solutions (MMS) problem to use.
    \descriptionend
    %---------------------------------------------------------------------------
    mms\_init             & \typflt & 0.1  & \slxtl &
    \descriptionbegin
    Initializes the solution to $[\textsl{mms\_init} \times (\textrm{exact MMS
    solution})]$.
    \descriptionend
    %---------------------------------------------------------------------------
    mms\_output\_solution & \typlog & \fls & \slxtl &
    \descriptionbegin
    Logical flag to completely skip the solver and just output the exact MMS
    solution.
    \descriptionend
    %%%%%%%%%%%%%%%%%%%%%%%%%%%%%%%%%%%%%%%%%%%%%%%%%%%%%%%%%%%%%%%%%%%%%%%%%%%%

    %%%%%%%%%%%%%%%%%%%%%%%%%%%%%%%%%%%%%%%%%%%%%%%%%%%%%%%%%%%%%%%%%%%%%%%%%%%%
    %%%%%%%%%%%%%%%%%%%%%%%%%%%%%%%%%%%%%%%%%%%%%%%%%%%%%%%%%%%%%%%%%%%%%%%%%%%%
    %%%%%%%%%%%%%%%%%%%%%%%%%%%%%%%%%%%%%%%%%%%%%%%%%%%%%%%%%%%%%%%%%%%%%%%%%%%%
    \groupline{INITIALIZATION / TEST CASE}
    %===========================================================================
    itestcase & \typint & 9 & \slbsc &
    \begin{minipage}[t]{\linewidth}\begin{flushleft}
    Specifies the test-case\slash problem to solve, primarily affecting the
    initialization of the simulation.
    \begin{tabular}{ @{\qquad} r @{ = } p{0.845\linewidth} @{} }
    1 & Generic flow, initialize to reference conditions. \\
    2 & Channel flow (not working). \\
    4 & Laminar subsonic boundary layer, initialize to
        \newline reference conditions. \\
    5 & Diagonally propagating Shu version of the isentropic
        \newline Euler vortex. \\
    -5 & Stationary Shu version of the isentropic Euler vortex. \\
    6 & Isentropic Euler vortex with propagation direction
        \newline determined by \textsl{alpha\_aoaref}. \\
    -6 & Isentropic Euler vortex propagating only in the
        \newline $y$-direction. \\
    7 & Advection of a density profile using the linear
        \newline advection equation. \\
    8 & Advection of an entropy profile using the linear
        \newline advection equation. \\
    9 & Taylor-Green vortex problem. \\
    10 & Special initialization function for SMC000 nozzle
         \newline jet case. \\
    11 & Double Mach reflection (not working). \\
    12 & Infinite cylinder. \\
    13 & Freestream preservation
    \end{tabular}
    \end{flushleft}\end{minipage} \\ \minorline
    %%%%%%%%%%%%%%%%%%%%%%%%%%%%%%%%%%%%%%%%%%%%%%%%%%%%%%%%%%%%%%%%%%%%%%%%%%%%

    \groupline{RESTART OPTIONS}
    %===========================================================================
    load\_restart\_file                & \typlog & \fls        & \slbsc &
    \descriptionbegin
    Logical flag used to indicate whether or not to read in a restart file.
    \descriptionend
    %---------------------------------------------------------------------------
    restart\_interval                  & \typint & 0           & \slbsc &
    \descriptionbegin
    Number of time steps between writing restart files. \NOTE A restart
    file is always written after completing the last time step.
    \descriptionend
    %---------------------------------------------------------------------------
    use\_old\_restart\_format          & \typlog & \fls        & \sldev &
    \descriptionbegin
    Logical flag to indicate that the restart file is in an older format used by
    an early conceptual version of GFR. \newline This should NEVER be used!
    \descriptionend
    %---------------------------------------------------------------------------
    restart\_file                      & \tc     & ``out.rst'' & \slbsc &
    \descriptionbegin
    File path to the solution restart file that will be read if \newline
    \textsl{load\_restart\_file} is true.
    \descriptionend
    %---------------------------------------------------------------------------
    restart\_output\_uses\_host\_roots & \typlog & \tru        & \slxpt &
    \descriptionbegin
    Logical flag that indicates whether all MPI processes are involved in
    writing to the restart file or if only the root processes on each host node
    write to the restart file.
    \NOTE This is only applicable for parallel simulations run across
    multiple physical computers, \forexample multiple nodes on the Pleiades
    supercomputer.
    \NOTE Only change this if you encounter MPI errors that occur when
    trying to write restart files. Changing this to false will most likely
    result in higher I\slash O file contention and possibly increased memory
    usage.
    \descriptionend
    %%%%%%%%%%%%%%%%%%%%%%%%%%%%%%%%%%%%%%%%%%%%%%%%%%%%%%%%%%%%%%%%%%%%%%%%%%%%

    %%%%%%%%%%%%%%%%%%%%%%%%%%%%%%%%%%%%%%%%%%%%%%%%%%%%%%%%%%%%%%%%%%%%%%%%%%%%
    %%%%%%%%%%%%%%%%%%%%%%%%%%%%%%%%%%%%%%%%%%%%%%%%%%%%%%%%%%%%%%%%%%%%%%%%%%%%
    %%%%%%%%%%%%%%%%%%%%%%%%%%%%%%%%%%%%%%%%%%%%%%%%%%%%%%%%%%%%%%%%%%%%%%%%%%%%
    \groupline{STATISTICAL OUTPUT OPTIONS}
    %===========================================================================
    iter\_out\_interval & \typint & 1 & \slbsc &
    \descriptionbegin
    Number of time steps between writing time-stepping and residual\slash
    convergence statistics to standard output. \NOTE When running the
    Taylor-Green vortex problem, this is also used as the interval for writing
    integrated kinetic energy dissipation rate (KEDR) statistics to the file
    ``TaylorGreen\_KEDR.dat''.
    \NOTE If $= 0$, $\qquad\;\;\textsl{iter\_out\_interval} = 1$ 
    \setlength{\abovedisplayskip}{-8pt}
    \setlength{\parindent}{\widthof{NOTE: }}
    \newline \indent else if $< 0$, \[\qquad\qquad\qquad
    \textsl{iter\_out\_interval} =
    \frac{\textsl{num\_timesteps}}{\left|\textsl{iter\_out\_interval}\right|}\]
    \setlength{\parindent}{0pt}
    \setlength{\abovedisplayskip}{4pt}
    \descriptionend
    %---------------------------------------------------------------------------
    results\_interval   & \typint & 0 & \slbsc &
    \descriptionbegin
    Number of time steps between writing residual\slash convergence statistics
    for all conserved and primitive variables to the file ``results.dat''.
    \setlength{\abovedisplayskip}{-8pt}
    \setlength{\belowdisplayskip}{0pt}
    \NOTE If $= 0$, \[\qquad\qquad\;\;\;\;\textsl{results\_interval} =
    \frac{\textsl{num\_timesteps}}{500}\]
    \setlength{\belowdisplayskip}{4pt}
    \setlength{\parindent}{\widthof{NOTE: }}
    \newline \indent else if $< 0$, \[\qquad\qquad\qquad
    \textsl{results\_interval} =
    \frac{\textsl{num\_timesteps}}{\left|\textsl{results\_interval}\right|}\]
    \setlength{\parindent}{0pt}
    \setlength{\abovedisplayskip}{4pt}
    \descriptionend
    %%%%%%%%%%%%%%%%%%%%%%%%%%%%%%%%%%%%%%%%%%%%%%%%%%%%%%%%%%%%%%%%%%%%%%%%%%%%

    %%%%%%%%%%%%%%%%%%%%%%%%%%%%%%%%%%%%%%%%%%%%%%%%%%%%%%%%%%%%%%%%%%%%%%%%%%%%
    %%%%%%%%%%%%%%%%%%%%%%%%%%%%%%%%%%%%%%%%%%%%%%%%%%%%%%%%%%%%%%%%%%%%%%%%%%%%
    %%%%%%%%%%%%%%%%%%%%%%%%%%%%%%%%%%%%%%%%%%%%%%%%%%%%%%%%%%%%%%%%%%%%%%%%%%%%
    \groupline{SOLUTION OUTPUT OPTIONS}
    %===========================================================================
    output\_dir        & \tc     & ``.'' & \slbsc &
    \descriptionbegin
    File path to the directory used for writing output files. \NOTE If
    one is available, it is recommended that the output directory be located on
    a Lustre file system in order to get the best I\slash O performance
    possible.
    \descriptionend
    %---------------------------------------------------------------------------
    output\_interval   & \typint & -5    & \slbsc &
    \descriptionbegin
    Number of time steps between writing CGNS solution files.
    \descriptionend
    %---------------------------------------------------------------------------
    continuous\_output & \typlog & \fls  & \slxtl &
    \descriptionbegin
    Logical flag to enable\slash disable the post-processor that attempts to
    create a smooth, continuous solution across all cells before writing it to
    the CGNS solution file.
    \descriptionend
    %---------------------------------------------------------------------------
    output\_order      & \typint & -3    & \sladv &
    \descriptionbegin
    Specifies the degree of the polynomial space used to represent the solution
    in the CGNS solution files. This simply allows the ability to over-sample
    the solution polynomial which provides a smoother solution within each cells
    when visualizing the solution.  \NOTE This is only used if
    $\textsl{output\_order}>\textsl{solution\_order}$.
    \descriptionend
    %---------------------------------------------------------------------------
    loc\_output\_pts   & \typint & 0     & \slxpt &
    \begin{minipage}[t]{\linewidth}\begin{flushleft}
    Specifies the location of the nodal points within the grid cells when
    writing the solution to the CGNS solution files.
    \begin{tabular}{ @{\qquad} r @{ = } p{0.85\linewidth} @{} }
    0 & Equi-distant nodes \\
    1 & Legendre-Gauss nodes \\
    2 & Legendre-Gauss-Lobatto nodes
    \end{tabular}
    \NoNewlineNOTE Using Legendre-Gauss nodes for the CGNS solution files will
    result in empty space between all grid cells because there is no
    connectivity information between cells.
    \end{flushleft}\end{minipage} \\ \minorline
    %%%%%%%%%%%%%%%%%%%%%%%%%%%%%%%%%%%%%%%%%%%%%%%%%%%%%%%%%%%%%%%%%%%%%%%%%%%%

    %%%%%%%%%%%%%%%%%%%%%%%%%%%%%%%%%%%%%%%%%%%%%%%%%%%%%%%%%%%%%%%%%%%%%%%%%%%%
    %%%%%%%%%%%%%%%%%%%%%%%%%%%%%%%%%%%%%%%%%%%%%%%%%%%%%%%%%%%%%%%%%%%%%%%%%%%%
    %%%%%%%%%%%%%%%%%%%%%%%%%%%%%%%%%%%%%%%%%%%%%%%%%%%%%%%%%%%%%%%%%%%%%%%%%%%%
    \groupline{GOVERNING EQUATIONS}
    %===========================================================================
    governing\_equations & \typint & 2 & \slbsc &
    \begin{minipage}[t]{\linewidth}\begin{flushleft}
    Specifies the governing equations to use.
    \begin{tabular}{ @{\qquad} r @{ = } p{0.85\linewidth} @{} }
    1 & Euler equations \\
    2 & Navier-Stokes equations
    \end{tabular}
    \end{flushleft}\end{minipage} \\ \minorline
    %---------------------------------------------------------------------------
    turbulence\_model    & \typint & 0 & \slnwk &
    \begin{minipage}[t]{\linewidth}\begin{flushleft}
    Specifies the turbulence model to use.
    \begin{tabular}{ @{\qquad} r @{ = } p{0.85\linewidth} @{} }
    0 & Laminar\slash Implicit LES \\
    1 & 1998 $k$-$\omega$ (not working) \\
    -1 & 2006 $k$-$\omega$ (not working)
    \end{tabular}
    \end{flushleft}\end{minipage} \\ \minorline
    %%%%%%%%%%%%%%%%%%%%%%%%%%%%%%%%%%%%%%%%%%%%%%%%%%%%%%%%%%%%%%%%%%%%%%%%%%%%

    %%%%%%%%%%%%%%%%%%%%%%%%%%%%%%%%%%%%%%%%%%%%%%%%%%%%%%%%%%%%%%%%%%%%%%%%%%%%
    %%%%%%%%%%%%%%%%%%%%%%%%%%%%%%%%%%%%%%%%%%%%%%%%%%%%%%%%%%%%%%%%%%%%%%%%%%%%
    %%%%%%%%%%%%%%%%%%%%%%%%%%%%%%%%%%%%%%%%%%%%%%%%%%%%%%%%%%%%%%%%%%%%%%%%%%%%
    \groupline{NONDIMENSIONAL REFERENCE CONDITIONS}
    %===========================================================================
    machref & \typflt & 0.4  & \slbsc &
    \descriptionbegin
    Reference Mach number.
    \descriptionend
    %---------------------------------------------------------------------------
    reyref  & \typflt & -1.0 & \slbsc &
    \descriptionbegin
    Reference Reynolds number (default evaluates to 8.7319e+6).
    \descriptionend
    %---------------------------------------------------------------------------
    gam     & \typflt & 1.4  & \slbsc &
    \descriptionbegin
    Reference $\gamma$ (ratio of specific heats).
    \descriptionend
    %%%%%%%%%%%%%%%%%%%%%%%%%%%%%%%%%%%%%%%%%%%%%%%%%%%%%%%%%%%%%%%%%%%%%%%%%%%%

    %%%%%%%%%%%%%%%%%%%%%%%%%%%%%%%%%%%%%%%%%%%%%%%%%%%%%%%%%%%%%%%%%%%%%%%%%%%%
    %%%%%%%%%%%%%%%%%%%%%%%%%%%%%%%%%%%%%%%%%%%%%%%%%%%%%%%%%%%%%%%%%%%%%%%%%%%%
    %%%%%%%%%%%%%%%%%%%%%%%%%%%%%%%%%%%%%%%%%%%%%%%%%%%%%%%%%%%%%%%%%%%%%%%%%%%%
    \groupline{TOTAL REFERENCE CONDITIONS}
    %===========================================================================
    ptotref       & \typflt & -1.0 & \slnwk &
    \descriptionbegin
    Reference total pressure (units of Pa).
    \descriptionend
    %---------------------------------------------------------------------------
    ttotref       & \typflt & -1.0 & \slnwk &
    \descriptionbegin
    Reference total temperature (units of Kelvin).
    \descriptionend
    %---------------------------------------------------------------------------
    ptot2p\_ratio & \typflt & -1.0 & \slnwk &
    \descriptionbegin
    Ratio of reference total pressure to reference static pressure.
    \descriptionend
    %%%%%%%%%%%%%%%%%%%%%%%%%%%%%%%%%%%%%%%%%%%%%%%%%%%%%%%%%%%%%%%%%%%%%%%%%%%%

    %%%%%%%%%%%%%%%%%%%%%%%%%%%%%%%%%%%%%%%%%%%%%%%%%%%%%%%%%%%%%%%%%%%%%%%%%%%%
    %%%%%%%%%%%%%%%%%%%%%%%%%%%%%%%%%%%%%%%%%%%%%%%%%%%%%%%%%%%%%%%%%%%%%%%%%%%%
    %%%%%%%%%%%%%%%%%%%%%%%%%%%%%%%%%%%%%%%%%%%%%%%%%%%%%%%%%%%%%%%%%%%%%%%%%%%%
    \groupline{REFERENCE CONDITIONS}
    %===========================================================================
    rhoref  & \typflt & -1.0     & \slbsc &
    \descriptionbegin
    Reference static density (default evaluates to 1.160833
    $\frac{\textrm{kg}}{\textrm{m}^3}$).
    \descriptionend
    %---------------------------------------------------------------------------
    tref    & \typflt & 300.0    & \slbsc &
    \descriptionbegin
    Reference static temperature (units of Kelvin).
    \descriptionend
    %---------------------------------------------------------------------------
    pref    & \typflt & 100000.0 & \slbsc &
    \descriptionbegin
    Reference static pressure (units of Pa).
    \descriptionend
    %---------------------------------------------------------------------------
    rgasref & \typflt & 287.15   & \slbsc &
    \descriptionbegin
    Reference gas constant (units of
    $\frac{\textrm{m}^2}{\textrm{s}^2\textrm{K}}$).
    \descriptionend
    %%%%%%%%%%%%%%%%%%%%%%%%%%%%%%%%%%%%%%%%%%%%%%%%%%%%%%%%%%%%%%%%%%%%%%%%%%%%

    %%%%%%%%%%%%%%%%%%%%%%%%%%%%%%%%%%%%%%%%%%%%%%%%%%%%%%%%%%%%%%%%%%%%%%%%%%%%
    %%%%%%%%%%%%%%%%%%%%%%%%%%%%%%%%%%%%%%%%%%%%%%%%%%%%%%%%%%%%%%%%%%%%%%%%%%%%
    %%%%%%%%%%%%%%%%%%%%%%%%%%%%%%%%%%%%%%%%%%%%%%%%%%%%%%%%%%%%%%%%%%%%%%%%%%%%
    \groupline{REFERENCE TURBULENCE CONDITIONS}
    %===========================================================================
    tinref & \typflt & 0.0    & \slnwk &
    \descriptionbegin
    Reference turbulence intensity. (not working)
    \descriptionend
    %---------------------------------------------------------------------------
    tvrref & \typflt & 1.0e-6 & \slnwk &
    \descriptionbegin
    Reference ratio of eddy viscosity to molecular viscosity. (not working)
    \descriptionend
    %%%%%%%%%%%%%%%%%%%%%%%%%%%%%%%%%%%%%%%%%%%%%%%%%%%%%%%%%%%%%%%%%%%%%%%%%%%%

    %%%%%%%%%%%%%%%%%%%%%%%%%%%%%%%%%%%%%%%%%%%%%%%%%%%%%%%%%%%%%%%%%%%%%%%%%%%%
    %%%%%%%%%%%%%%%%%%%%%%%%%%%%%%%%%%%%%%%%%%%%%%%%%%%%%%%%%%%%%%%%%%%%%%%%%%%%
    %%%%%%%%%%%%%%%%%%%%%%%%%%%%%%%%%%%%%%%%%%%%%%%%%%%%%%%%%%%%%%%%%%%%%%%%%%%%
    \groupline{REFERENCE ANGLES OF ATTACK}
    %===========================================================================
    alpha\_aoaref & \typflt & 0.0 & \slbsc &
    \descriptionbegin
    Reference angle of attack in the $x$-$y$ plane.
    \descriptionend
    %---------------------------------------------------------------------------
    beta\_aoaref  & \typflt & 0.0 & \slbsc &
    \descriptionbegin
    Reference angle of attack in the $x$-$z$ plane.
    \descriptionend
    %%%%%%%%%%%%%%%%%%%%%%%%%%%%%%%%%%%%%%%%%%%%%%%%%%%%%%%%%%%%%%%%%%%%%%%%%%%%

    %%%%%%%%%%%%%%%%%%%%%%%%%%%%%%%%%%%%%%%%%%%%%%%%%%%%%%%%%%%%%%%%%%%%%%%%%%%%
    %%%%%%%%%%%%%%%%%%%%%%%%%%%%%%%%%%%%%%%%%%%%%%%%%%%%%%%%%%%%%%%%%%%%%%%%%%%%
    %%%%%%%%%%%%%%%%%%%%%%%%%%%%%%%%%%%%%%%%%%%%%%%%%%%%%%%%%%%%%%%%%%%%%%%%%%%%
    \groupline{LAMINAR AND TURBULENT PRANDTL NUMBERS}
    %===========================================================================
    Pr  & \typflt & 0.72 & \slbsc &
    \descriptionbegin
    Laminar Prandtl number.
    \descriptionend
    %---------------------------------------------------------------------------
    Prt & \typflt & 0.72 & \slnwk &
    \descriptionbegin
    Turbulent Prandtl number. (not working)
    \descriptionend
    %%%%%%%%%%%%%%%%%%%%%%%%%%%%%%%%%%%%%%%%%%%%%%%%%%%%%%%%%%%%%%%%%%%%%%%%%%%%

    %%%%%%%%%%%%%%%%%%%%%%%%%%%%%%%%%%%%%%%%%%%%%%%%%%%%%%%%%%%%%%%%%%%%%%%%%%%%
    %%%%%%%%%%%%%%%%%%%%%%%%%%%%%%%%%%%%%%%%%%%%%%%%%%%%%%%%%%%%%%%%%%%%%%%%%%%%
    %%%%%%%%%%%%%%%%%%%%%%%%%%%%%%%%%%%%%%%%%%%%%%%%%%%%%%%%%%%%%%%%%%%%%%%%%%%%
    \groupline{SUTHERLANDS LAW COEFFICIENTS}
    %===========================================================================
    suth\_muref & \typflt & 1.716e-5 & \slbsc &
    \descriptionbegin
    Constant $\mu_0$ in Sutherland's law.
    \descriptionend
    %---------------------------------------------------------------------------
    suth\_Tref  & \typflt & 273.0    & \slbsc &
    \descriptionbegin
    Constant $T_0$ in Sutherland's law.
    \descriptionend
    %---------------------------------------------------------------------------
    suth\_Sref  & \typflt & 110.4    & \slbsc &
    \descriptionbegin
    Constant $S$ in Sutherland's law.
    \descriptionend
    %%%%%%%%%%%%%%%%%%%%%%%%%%%%%%%%%%%%%%%%%%%%%%%%%%%%%%%%%%%%%%%%%%%%%%%%%%%%

    %%%%%%%%%%%%%%%%%%%%%%%%%%%%%%%%%%%%%%%%%%%%%%%%%%%%%%%%%%%%%%%%%%%%%%%%%%%%
    %%%%%%%%%%%%%%%%%%%%%%%%%%%%%%%%%%%%%%%%%%%%%%%%%%%%%%%%%%%%%%%%%%%%%%%%%%%%
    %%%%%%%%%%%%%%%%%%%%%%%%%%%%%%%%%%%%%%%%%%%%%%%%%%%%%%%%%%%%%%%%%%%%%%%%%%%%
    \groupline{FILTER OPTIONS}
    %===========================================================================
    Filter\_Option & \typint & 0    & \sladv &
    \begin{minipage}[t]{\linewidth}\begin{flushleft}
    Specifies the type of standard filter to use for stabilization.
    \begin{tabular}{ @{\quad} r @{ $\Rightarrow$ } p{0.85\linewidth} @{} }
    $= 0$ & The standard filter is not used. \\
    \multicolumn{2}{ @{} l @{} }{\textbf{For Structured Elements}} \\
    $= 1$ & \begin{minipage}[t]{\linewidth}\begin{flushleft}
    Use a tensor product of the 1D exponential filter which amounts to the
    1D filter being applied separately for each direction where \newline
    $N=\textsl{solution\_order}\qquad$ for 1D, 2D, and 3D
    \end{flushleft}\end{minipage} \\
    $= 2$ & \begin{minipage}[t]{\linewidth}\begin{flushleft}
    Use a true tensor product of the 1D exponential filter where \newline
    $N = 2 \times \textsl{solution\_order}\qquad$ for 2D \newline
    $N = 3 \times \textsl{solution\_order}\qquad$ for 3D
    \end{flushleft}\end{minipage} \\
    \multicolumn{2}{ @{} l @{} }{\textbf{For Unstructured Elements}} \\
    $\neq0$ & \begin{minipage}[t]{\linewidth}\begin{flushleft}
    Use the exponential filter where \newline
    $N=\textsl{solution\_order}\qquad$ for 1D, 2D, and 3D
    \end{flushleft}\end{minipage} \\
    \end{tabular} \newline
    \newline \newline
    The exponential filter is defined as: \newline \newline
    \begin{tabular}{ @{\quad} r @{ = } p{0.85\linewidth} @{} }
    $N_c$ & filter cutoff and must have a value  $\;0 \leq N_c \leq N$ \\
    $\eta_c$ & $\dfrac{\textrm{real}(N_c)}{\textrm{real}(N)}$
    \end{tabular} \newline \newline \newline
    \setlength{\parindent}{12pt}
    $\texttt{filter\_diagonal}(0\!:\!N) = 0.0$ \newline \newline
    $\texttt{do}\;i = N_c , N$ \newline
    \indent $\eta = \dfrac{\texttt{real}(i)}{\texttt{real}(N)}$ \newline
    \indent $\texttt{filter\_diagonal}(i) = (1.0 - \varepsilon)
    \exp\left[ -\alpha \left(\dfrac{\eta - \eta_c}{1.0 -
    \eta_c}\right)^{\textstyle{s}}
    \right]$ \newline
    $\texttt{end do}$ \newline
    \setlength{\parindent}{0pt}
    \end{flushleft}\end{minipage} \\ \minorline
    %---------------------------------------------------------------------------
    Filter\_etac   & \typflt & 0.0  & \sladv &
    \descriptionbegin
    Modal cutoff ($\eta_c$) for the standard filter, below which the lower modes
    are left untouched. \NOTE The value of \textsl{Filter\_etac} must be between
    0 and 1.
    \descriptionend
    %---------------------------------------------------------------------------
    Filter\_Order  & \typint & 16   & \sladv &
    \descriptionbegin
    Order ($s$) of the exponential filter.
    \descriptionend
    %---------------------------------------------------------------------------
    Filter\_Alpha  & \typflt & 36.0 & \sladv &
    \descriptionbegin
    Maximum damping parameter ($\alpha$) for the standard parameter.
    \descriptionend
    %---------------------------------------------------------------------------
    Filter\_eps    & \typflt & 0.0  & \sladv &
    \descriptionbegin
    Leading damping parameter ($\varepsilon$) for the standard filter.
    \descriptionend
    %%%%%%%%%%%%%%%%%%%%%%%%%%%%%%%%%%%%%%%%%%%%%%%%%%%%%%%%%%%%%%%%%%%%%%%%%%%%

    %%%%%%%%%%%%%%%%%%%%%%%%%%%%%%%%%%%%%%%%%%%%%%%%%%%%%%%%%%%%%%%%%%%%%%%%%%%%
    %%%%%%%%%%%%%%%%%%%%%%%%%%%%%%%%%%%%%%%%%%%%%%%%%%%%%%%%%%%%%%%%%%%%%%%%%%%%
    %%%%%%%%%%%%%%%%%%%%%%%%%%%%%%%%%%%%%%%%%%%%%%%%%%%%%%%%%%%%%%%%%%%%%%%%%%%%
    \groupline{LIMITER FILTER OPTIONS}
    %===========================================================================
    Limiter\_Filter\_Option & \typint & 0    & \sladv &
    \descriptionbegin
    The details for this are the same as for \textsl{Filter\_Option}, but this
    is a second filter that is only used as a limiter on any cells that have
    been marked as a troubled cell.
    \NOTE The idea is you can use \textsl{Filter\_Option} as a very weak filter
    that is constantly being applied regardless of whether or not the solution
    is oscillatory within the cell, and you can use
    \textsl{Limiter\_Filter\_Option} as a stronger filter that is only applied
    to cells that have been marked as a troubled cell to prevent the oscillatory
    solutions in these cells from causing the simulation to diverge.
    \descriptionend
    %---------------------------------------------------------------------------
    Limiter\_Filter\_etac   & \typflt & 0.0  & \sladv &
    \descriptionbegin
    Same as \textsl{Filter\_etac} but for the limiter filter.
    \descriptionend
    %---------------------------------------------------------------------------
    Limiter\_Filter\_Order  & \typint & 16   & \sladv &
    \descriptionbegin
    Same as \textsl{Filter\_Order} but for the limiter filter.
    \descriptionend
    %---------------------------------------------------------------------------
    Limiter\_Filter\_Alpha  & \typflt & 36.0 & \sladv &
    \descriptionbegin
    Same as \textsl{Filter\_Alpha} but for the limiter filter.
    \descriptionend
    %---------------------------------------------------------------------------
    Limiter\_Filter\_eps    & \typflt & 0.0  & \sladv &
    \descriptionbegin
    Same as \textsl{Filter\_eps} but for the limiter filter.
    \descriptionend
    %%%%%%%%%%%%%%%%%%%%%%%%%%%%%%%%%%%%%%%%%%%%%%%%%%%%%%%%%%%%%%%%%%%%%%%%%%%%

    %%%%%%%%%%%%%%%%%%%%%%%%%%%%%%%%%%%%%%%%%%%%%%%%%%%%%%%%%%%%%%%%%%%%%%%%%%%%
    %%%%%%%%%%%%%%%%%%%%%%%%%%%%%%%%%%%%%%%%%%%%%%%%%%%%%%%%%%%%%%%%%%%%%%%%%%%%
    %%%%%%%%%%%%%%%%%%%%%%%%%%%%%%%%%%%%%%%%%%%%%%%%%%%%%%%%%%%%%%%%%%%%%%%%%%%%
    \groupline{LIMITER OPTIONS}
    %===========================================================================
    Limiter\_Option & \typint & 0 & \sladv &
    \begin{minipage}[t]{\linewidth}\begin{flushleft}
    Specifies the type of limiter to use.
    \begin{tabular}{ @{\quad} l @{ $\Rightarrow$ } p{0.85\linewidth} @{} }
    $= 0$ & Do not use a limiter. \\
    $= -1$ & Use the resolution indicator to identify troubled cells. For any
    troubled cells, project the solution down to a order that is dynamically
    chosen by the value of the resolution indicator; cells with higher indicator
    values will be projected to lower orders, whereas cells with lower indicator
    values will be projected down only an order or two from the solution order.
    \\
    $= 1$ & Use the resolution indicator to identify troubled cells. The
    resolution indicator is based on the integral of the (P-1)-order solution
    density divided by the (P)-order solution density.\\
    $= 2$ & Use the jump indicator to identify troubled cells. The jump
    indicator is based on the pressure difference with neighboring cells for all
    the faces of a cell. \\
    \end{tabular}
    \newline \newline
    If $\textsl{Limiter\_Filter\_Option} = 0$, the limiting will be performed by
    projecting the solution in each troubled cell down by one order, after which
    all troubled cells will be rechecked to see if they are still troubled. If
    any are still troubled the solution in those cells will be projected down
    another order, and this loop continues until there are no more troubled
    cells or the remaining troubled cells have been reduced to P1 solutions.
    \newline \newline
    If $\textsl{Limiter\_Filter\_Option} \neq 0$, the limiter filter will be
    applied once to all troubled cells.
    \newline
    \end{flushleft}\end{minipage} \\ \minorline
    %%%%%%%%%%%%%%%%%%%%%%%%%%%%%%%%%%%%%%%%%%%%%%%%%%%%%%%%%%%%%%%%%%%%%%%%%%%%

    %%%%%%%%%%%%%%%%%%%%%%%%%%%%%%%%%%%%%%%%%%%%%%%%%%%%%%%%%%%%%%%%%%%%%%%%%%%%
    %%%%%%%%%%%%%%%%%%%%%%%%%%%%%%%%%%%%%%%%%%%%%%%%%%%%%%%%%%%%%%%%%%%%%%%%%%%%
    %%%%%%%%%%%%%%%%%%%%%%%%%%%%%%%%%%%%%%%%%%%%%%%%%%%%%%%%%%%%%%%%%%%%%%%%%%%%
    \groupline{CONTROL OVER FREQUENCY OF FILTERING/LIMITING}
    %===========================================================================
    Filtering\_Interval & \typint & 0 & \sladv &
    \begin{minipage}[t]{\linewidth}\begin{flushleft}
    Specifies the frequency to apply the standard filter.
    \begin{tabular}{ @{\quad} r @{ $\Rightarrow$ } p{0.85\linewidth} @{} }
    $= 0$ & Apply the standard filter after each Runge-Kutta stage. \\
   %$> 0$ & Apply the standard filter after the last Runge-Kutta stage of every
   %$\textsl{Filtering\_Interval}^{\;\textrm{th}}$ time step. \\
    $> 0$ & \begin{minipage}[t]{\linewidth}\begin{flushleft}
            Apply the standard filter after the last Runge-Kutta stage if
            \newline $\textrm{mod} \left( \textsl{Filtering\_Interval},
            \textsl{current\_time\_step} \right) \eql 0$.
            \end{flushleft}\end{minipage} \\
    $< 0$ & The standard filter will NOT be applied even if \newline
    $\textsl{Filter\_Option}\neq0$
    \end{tabular}
    \end{flushleft}\end{minipage} \\ \minorline
    %---------------------------------------------------------------------------
    Limiting\_Interval  & \typint & 0 & \sladv &
    \begin{minipage}[t]{\linewidth}\begin{flushleft}
    Specifies the frequency to apply the limiter.
    \begin{tabular}{ @{\quad} r @{ $\Rightarrow$ } p{0.85\linewidth} @{} }
    $= 0$ & Apply the limiter after each Runge-Kutta stage. \\
    $> 0$ & \begin{minipage}[t]{\linewidth}\begin{flushleft}
            Apply the limiter after the last Runge-Kutta stage if
            \newline $\textrm{mod} \left( \textsl{Limiting\_Interval},
            \textsl{current\_time\_step} \right) \eql 0$.
            \end{flushleft}\end{minipage} \\
    $< 0$ & The limiter will NOT be applied even if \newline
    $\textsl{Limiter\_Option}\neq0$
    \end{tabular}
    \end{flushleft}\end{minipage} \\ \minorline
    %%%%%%%%%%%%%%%%%%%%%%%%%%%%%%%%%%%%%%%%%%%%%%%%%%%%%%%%%%%%%%%%%%%%%%%%%%%%

    %%%%%%%%%%%%%%%%%%%%%%%%%%%%%%%%%%%%%%%%%%%%%%%%%%%%%%%%%%%%%%%%%%%%%%%%%%%%
    %%%%%%%%%%%%%%%%%%%%%%%%%%%%%%%%%%%%%%%%%%%%%%%%%%%%%%%%%%%%%%%%%%%%%%%%%%%%
    %%%%%%%%%%%%%%%%%%%%%%%%%%%%%%%%%%%%%%%%%%%%%%%%%%%%%%%%%%%%%%%%%%%%%%%%%%%%
    \groupline{ISENTROPIC VORTEX CONSTANTS}
    %===========================================================================
    vortex\_bigR     & \typflt & 1.0  & \slspc &
    \descriptionbegin
    Constant used for the isentropic Euler vortex problem.
    \descriptionend
    %---------------------------------------------------------------------------
    vortex\_bigGam   & \typflt & 5.0  & \slspc &
    \descriptionbegin
    Constant used for the isentropic Euler vortex problem.
    \descriptionend
    %---------------------------------------------------------------------------
    VortexGrid\_Lref & \typflt & 10.0 & \slspc &
    \descriptionbegin
    Reference length for the grid domain, \thatis the grid is a square with
    bounds of
    $\left[-\textsl{VortexGrid\_Lref},+\textsl{VortexGrid\_Lref}\right]$ in both
    the $x$- and $y$-directions.
    \descriptionend
    %%%%%%%%%%%%%%%%%%%%%%%%%%%%%%%%%%%%%%%%%%%%%%%%%%%%%%%%%%%%%%%%%%%%%%%%%%%%

    %%%%%%%%%%%%%%%%%%%%%%%%%%%%%%%%%%%%%%%%%%%%%%%%%%%%%%%%%%%%%%%%%%%%%%%%%%%%
    %%%%%%%%%%%%%%%%%%%%%%%%%%%%%%%%%%%%%%%%%%%%%%%%%%%%%%%%%%%%%%%%%%%%%%%%%%%%
    %%%%%%%%%%%%%%%%%%%%%%%%%%%%%%%%%%%%%%%%%%%%%%%%%%%%%%%%%%%%%%%%%%%%%%%%%%%%
    \groupline{TAYLOR-GREEN VORTEX CONSTANTS}
    %===========================================================================
    VortexGrid\_DOF              & \typint & 48   & \slspc &
    \descriptionbegin
    Number of solution points \slash degrees-of-freedom (DoF) in each direction
    to use for the isentropic Euler vortex and Taylor-Green vortex problems.
    \descriptionend
    %---------------------------------------------------------------------------
    VortexGrid\_random\_perturb  & \typlog & \fls & \slspc &
    \descriptionbegin
    Logical flag for applying a random perturbation to all the interior grid
    nodes for the isentropic Euler vortex and Taylor-Green vortex problems.
    \descriptionend
    %---------------------------------------------------------------------------
    VortexGrid\_perturb\_percent & \typint & 30   & \slspc &
    \descriptionbegin
    Percent of the uniform grid cell length to perturb the interior grid nodes
    if \textsl{VortexGrid\_random\_perturb} is true.
    \descriptionend
    %%%%%%%%%%%%%%%%%%%%%%%%%%%%%%%%%%%%%%%%%%%%%%%%%%%%%%%%%%%%%%%%%%%%%%%%%%%%

    %%%%%%%%%%%%%%%%%%%%%%%%%%%%%%%%%%%%%%%%%%%%%%%%%%%%%%%%%%%%%%%%%%%%%%%%%%%%
    %%%%%%%%%%%%%%%%%%%%%%%%%%%%%%%%%%%%%%%%%%%%%%%%%%%%%%%%%%%%%%%%%%%%%%%%%%%%
    %%%%%%%%%%%%%%%%%%%%%%%%%%%%%%%%%%%%%%%%%%%%%%%%%%%%%%%%%%%%%%%%%%%%%%%%%%%%
    \groupline{CHANNEL FLOW PARAMETERS}
    %===========================================================================
    channel\_body\_force & \typflt & 0.0   & \slnwk &
    \descriptionbegin
    Body force used to drive the channel flow problem. (not working)
    \descriptionend
    %---------------------------------------------------------------------------
    channel\_init\_file  & \tc     & `` '' & \slnwk &
    \descriptionbegin
    File path to the file used to initialize the solution for the channel flow
    problem. (not working)
    \descriptionend
    %%%%%%%%%%%%%%%%%%%%%%%%%%%%%%%%%%%%%%%%%%%%%%%%%%%%%%%%%%%%%%%%%%%%%%%%%%%%

    %%%%%%%%%%%%%%%%%%%%%%%%%%%%%%%%%%%%%%%%%%%%%%%%%%%%%%%%%%%%%%%%%%%%%%%%%%%%
    %%%%%%%%%%%%%%%%%%%%%%%%%%%%%%%%%%%%%%%%%%%%%%%%%%%%%%%%%%%%%%%%%%%%%%%%%%%%
    %%%%%%%%%%%%%%%%%%%%%%%%%%%%%%%%%%%%%%%%%%%%%%%%%%%%%%%%%%%%%%%%%%%%%%%%%%%%
    \groupline{OPTIONS FOR MODIFYING BC ALGORITHMS}
    %===========================================================================
    walls\_are\_exact   & \typlog & \tru & \sladv &
    \begin{minipage}[t]{\linewidth}\begin{flushleft}
    Logical flag that determines if the solution on wall boundary conditions are
    treated as either the exact flow conditions or the ghost state that equals
    the exact flow conditions when averaged with the interior state.
    \begin{tabular}{ @{\qquad} r @{ = } p{0.75\linewidth} @{} }
    .TRUE. & The wall boundary solution is the exact wall boundary condition,
    \forexample $\left[ u_{\textrm{wall}}, v_{\textrm{wall}}, w_{\textrm{wall}}
    \right] = 0$ for a no-slip wall. \\
    .FALSE. & The average of the wall boundary solution \newline
    and the interior solution on the boundary face \newline
    recovers the exact wall boundary condition, \newline
    \forexample $\left[ u_{\textrm{wall}}, v_{\textrm{wall}}, w_{\textrm{wall}}
    \right] = -\left[ u_{\textrm{wall}}, v_{\textrm{wall}}, w_{\textrm{wall}}
    \right]$ \newline for a no-slip wall.
    \end{tabular}
    \NoNewlineNOTE Hartmann (2014) reports that using exact wall boundary
    conditions seems to be more accurate on coarse grids and for lower values of
    \textsl{solution\_order}, whereas using the averaging method seems to be a
    little more stable. Details can be found at \newline 
    {\small \url{http://elib.dlr.de/90967/1/hartmann_leicht_VKI_LS_2014-3.pdf}}
    \end{flushleft}\end{minipage} \\ \minorline
    %---------------------------------------------------------------------------
    sub\_inflow\_method & \typint & 1    & \sladv &
    \begin{minipage}[t]{\linewidth}\begin{flushleft}
    Specifies which subsonic inflow boundary condition algorithm to use.
    \begin{tabular}{ @{\qquad} r @{ = } p{0.85\linewidth} @{} }
    1 & Hold the total pressure and total temperature constant at the inflow,
    use the outgoing characteristic from the interior to perform a Newton
    iteration to get the static temperature, speed of sound, and normal velocity
    magnitude for the exterior state. \\
    2 & Use the inflow static density and velocity, and the interior static
    pressure. \\
    3 & Hold the inflow total pressure and total temperature constant, and use
    the interior velocity. \\
    4 & Hold the inflow total pressure and total temperature constant, and use
    the interior static pressure.
    \end{tabular}
    \NoNewlineNOTE It is highly recommended to use option 1.
    \end{flushleft}\end{minipage} \\ \minorline
    %---------------------------------------------------------------------------
    use\_bc\_conflicts  & \typlog & \tru & \slspc &
    \descriptionbegin
    Logical flag for determining whether or not to overwrite boundary solutions
    in adjacent cells if co-located solution points have different boundary
    conditions. \NOTE This is only relevant if using
    Legendre-Gauss-Lobatto points for solution points.
    \descriptionend
    %%%%%%%%%%%%%%%%%%%%%%%%%%%%%%%%%%%%%%%%%%%%%%%%%%%%%%%%%%%%%%%%%%%%%%%%%%%%

    %%%%%%%%%%%%%%%%%%%%%%%%%%%%%%%%%%%%%%%%%%%%%%%%%%%%%%%%%%%%%%%%%%%%%%%%%%%%
    %%%%%%%%%%%%%%%%%%%%%%%%%%%%%%%%%%%%%%%%%%%%%%%%%%%%%%%%%%%%%%%%%%%%%%%%%%%%
    %%%%%%%%%%%%%%%%%%%%%%%%%%%%%%%%%%%%%%%%%%%%%%%%%%%%%%%%%%%%%%%%%%%%%%%%%%%%
    \groupline{CGNS OUTPUT OPTIONS}
    %===========================================================================
    cgns\_use\_queue                      & \typlog & \fls & \sldev &
    \descriptionbegin
    Flag enabling/disabling the CGNS queuing system when writing parallel CGNS
    solution files. \NOTE This is only applicable if using a version of
    the CGNS library older than 3.3.0, \forexample 3.2.1.
    \descriptionend
    %---------------------------------------------------------------------------
    interpolate\_before\_output\_variable & \typlog & \tru & \sldev &
    \descriptionbegin
    Flag determining whether the solution variables are computed before or after
    interpolating to the output points.
    \descriptionend
    %---------------------------------------------------------------------------
    convert\_restart\_to\_cgns            & \typlog & \fls & \slutl &
    \descriptionbegin
    Flag used to read in a restart file and immediately write a CGNS solution
    file, bypassing the solver entirely.
    \descriptionend
    %---------------------------------------------------------------------------
    completely\_disable\_cgns             & \typlog & \fls & \sladv &
    \descriptionbegin
    Flag used to completely disable CGNS solutions from being output at the
    output interval.
    \descriptionend
    %%%%%%%%%%%%%%%%%%%%%%%%%%%%%%%%%%%%%%%%%%%%%%%%%%%%%%%%%%%%%%%%%%%%%%%%%%%%

    %%%%%%%%%%%%%%%%%%%%%%%%%%%%%%%%%%%%%%%%%%%%%%%%%%%%%%%%%%%%%%%%%%%%%%%%%%%%
    %%%%%%%%%%%%%%%%%%%%%%%%%%%%%%%%%%%%%%%%%%%%%%%%%%%%%%%%%%%%%%%%%%%%%%%%%%%%
    %%%%%%%%%%%%%%%%%%%%%%%%%%%%%%%%%%%%%%%%%%%%%%%%%%%%%%%%%%%%%%%%%%%%%%%%%%%%
    \groupline{CONVERGENCE CRITERIA}
    %===========================================================================
    convergence\_order\_abs\_res & \typflt & -16.0 & \sladv &
    \descriptionbegin
    This is the convergence goal for the orders of magnitude of the absolute
    residual.
    \descriptionend
    %---------------------------------------------------------------------------
    convergence\_order\_max\_res & \typflt & -16.0 & \sladv &
    \descriptionbegin
    This is the convergence goal for the orders of magnitude reduction of the
    residual relative to the maximum value of the residual during the
    simulation.
    \descriptionend
    %%%%%%%%%%%%%%%%%%%%%%%%%%%%%%%%%%%%%%%%%%%%%%%%%%%%%%%%%%%%%%%%%%%%%%%%%%%%

    %%%%%%%%%%%%%%%%%%%%%%%%%%%%%%%%%%%%%%%%%%%%%%%%%%%%%%%%%%%%%%%%%%%%%%%%%%%%
    %%%%%%%%%%%%%%%%%%%%%%%%%%%%%%%%%%%%%%%%%%%%%%%%%%%%%%%%%%%%%%%%%%%%%%%%%%%%
    %%%%%%%%%%%%%%%%%%%%%%%%%%%%%%%%%%%%%%%%%%%%%%%%%%%%%%%%%%%%%%%%%%%%%%%%%%%%
    \groupline{PROFILE PARALLEL IO}
    %===========================================================================
    profile\_io\_all     & \typlog & \fls & \sldev &
    \descriptionbegin
    If true, sets both \textsl{profile\_io\_cgns} and
    \textsl{profile\_io\_restart} to true.
    \descriptionend
    %---------------------------------------------------------------------------
    profile\_io\_cgns    & \typlog & \fls & \sldev &
    \descriptionbegin
    Flag to activate some simple timers that time how long it takes to write
    each CGNS solution file, and output the results to standard output.
    \descriptionend
    %---------------------------------------------------------------------------
    profile\_io\_restart & \typlog & \fls & \sldev &
    \descriptionbegin
    Flag to activate some simple timers that time how long it takes to write
    each restart or time-averaged restart file, and output the results to
    standard output.
    \descriptionend
    %%%%%%%%%%%%%%%%%%%%%%%%%%%%%%%%%%%%%%%%%%%%%%%%%%%%%%%%%%%%%%%%%%%%%%%%%%%%

    %%%%%%%%%%%%%%%%%%%%%%%%%%%%%%%%%%%%%%%%%%%%%%%%%%%%%%%%%%%%%%%%%%%%%%%%%%%%
    %%%%%%%%%%%%%%%%%%%%%%%%%%%%%%%%%%%%%%%%%%%%%%%%%%%%%%%%%%%%%%%%%%%%%%%%%%%%
    %%%%%%%%%%%%%%%%%%%%%%%%%%%%%%%%%%%%%%%%%%%%%%%%%%%%%%%%%%%%%%%%%%%%%%%%%%%%
    \groupline{OPTIONS TO OUTPUT BOUNDARY PROFILE}
    %===========================================================================
    output\_bnd\_profiles & \typlog & \fls & \slxtl &
    \descriptionbegin
    Flag used to output the solution profiles on certain boundary faces (highly
    experimental).
    \descriptionend
    %%%%%%%%%%%%%%%%%%%%%%%%%%%%%%%%%%%%%%%%%%%%%%%%%%%%%%%%%%%%%%%%%%%%%%%%%%%%

    %%%%%%%%%%%%%%%%%%%%%%%%%%%%%%%%%%%%%%%%%%%%%%%%%%%%%%%%%%%%%%%%%%%%%%%%%%%%
    %%%%%%%%%%%%%%%%%%%%%%%%%%%%%%%%%%%%%%%%%%%%%%%%%%%%%%%%%%%%%%%%%%%%%%%%%%%%
    %%%%%%%%%%%%%%%%%%%%%%%%%%%%%%%%%%%%%%%%%%%%%%%%%%%%%%%%%%%%%%%%%%%%%%%%%%%%
    \groupline{CONTROL DOT\_PRODUCT PROCEDURES BOUND TO MATRIX DERIVED-TYPE}
    %===========================================================================
    use\_unrolled\_dot\_products & \typlog & \tru & \sldev &
    \descriptionbegin
    Flag for enabling/disabling the manually unrolled dot product functions.
    \descriptionend
    %%%%%%%%%%%%%%%%%%%%%%%%%%%%%%%%%%%%%%%%%%%%%%%%%%%%%%%%%%%%%%%%%%%%%%%%%%%%

    %%%%%%%%%%%%%%%%%%%%%%%%%%%%%%%%%%%%%%%%%%%%%%%%%%%%%%%%%%%%%%%%%%%%%%%%%%%%
    %%%%%%%%%%%%%%%%%%%%%%%%%%%%%%%%%%%%%%%%%%%%%%%%%%%%%%%%%%%%%%%%%%%%%%%%%%%%
    %%%%%%%%%%%%%%%%%%%%%%%%%%%%%%%%%%%%%%%%%%%%%%%%%%%%%%%%%%%%%%%%%%%%%%%%%%%%
    \groupline{LUSTRE FILE SYSTEM STRIPING PARAMETERS}
    %===========================================================================
    lustre\_stripe\_count & \typint & 64     & \slxpt &
    \descriptionbegin
    Specifies the approximate number of stripes to use when writing a CGNS
    solution file or a restart file. \NOTE This is only used if
    \textsl{output\_dir} is on a Lustre file system. \NOTE The code
    will take this value and find the closest integer value satisfying:
    $\quad \textrm{mod} \left( \textsl{ncpu} , \textsl{lustre\_stripe\_count}
    \right) \eql 0$
    \descriptionend
    %---------------------------------------------------------------------------
    lustre\_stripe\_size  & \tc[10] & ``4m'' & \slxpt &
    \descriptionbegin
    Specifies the number of bytes to store on each stripe before moving to the
    next stripe. \NOTE This is only used if \textsl{output\_dir} is on
    a Lustre file system. \NOTE The value of \textsl{lustre\_stripe\_size}
    must be a multiple of 65,536 bytes (64 KB), and the suffixes `k', `m', or
    `g' can be used to specify units of KB, MB, or GB, respectively.
    \EXAMPLE $\quad\textrm{``4m''} \Rightarrow 4~\textrm{MB} \Rightarrow
    4,194,304~\textrm{bytes}$
    \descriptionend
    %%%%%%%%%%%%%%%%%%%%%%%%%%%%%%%%%%%%%%%%%%%%%%%%%%%%%%%%%%%%%%%%%%%%%%%%%%%%

    %%%%%%%%%%%%%%%%%%%%%%%%%%%%%%%%%%%%%%%%%%%%%%%%%%%%%%%%%%%%%%%%%%%%%%%%%%%%
    %%%%%%%%%%%%%%%%%%%%%%%%%%%%%%%%%%%%%%%%%%%%%%%%%%%%%%%%%%%%%%%%%%%%%%%%%%%%
    %%%%%%%%%%%%%%%%%%%%%%%%%%%%%%%%%%%%%%%%%%%%%%%%%%%%%%%%%%%%%%%%%%%%%%%%%%%%
    \groupline{TIME AVERAGING}
    %===========================================================================
    output\_time\_averaging     & \typlog & \fls        & \slbsc &
    \descriptionbegin
    Logical flag that enables\slash disables the accumulation of time-averaged
    flow variables. If enabled, time-averaged restart and CGNS solution files
    will be written whenever the standard restart and CGNS solution files are
    written.  \NOTE In order to save disk space, the name of the time-averaged
    restart file will switch between
    \setlength{\parindent}{20pt}\newline\indent
    ``\textsl{output\_dir}\slash{Restart\_files}\slash{out.1.ave}''
    \newline and \newline\indent
    ``\textsl{output\_dir}\slash{Restart\_files}\slash{out.2.ave}''.
    \newline After successfully writing a new time-averaged restart file, a
    symbolic link \newline\indent
    ``\textsl{output\_dir}\slash{Restart\_files}\slash{out.current.ave}'' 
    \newline will be created that points to the latest time-averaged restart
    file.    
    \setlength{\parindent}{0pt}
    \descriptionend
    %---------------------------------------------------------------------------
    time\_ave\_file             & \tc     & ``out.ave'' & \slbsc &
    \descriptionbegin
    File path to the time-averaged solution restart file that will be read in if
    \textsl{load\_restart\_file} and \textsl{output\_time\_averaging} are both
    true. The code will add the accumulation of time-averaged flow variables
    from the new running simulation to the time-averaged flow variables of the
    previous simulation from which the new simulation restarted.
    \descriptionend
    %---------------------------------------------------------------------------
    time\_scaling\_factor       & \typflt & 1.0         & \slbsc &
    \descriptionbegin
    This is a scaling factor that is used to scale the value of the running time
    within the simulation that is written to standard output and the results
    file. This scaling has no affect on a simulation besides the cosmetic
    scaling of these values when they are output.
    \descriptionend
    %---------------------------------------------------------------------------
    time\_ave\_vel\_is\_axisymm & \typlog & \fls        & \sladv &
    \descriptionbegin
    Logical flag that determines what velocity components to compute for the
    time-averaged flow variables. If true, the grid is assumed to be
    axisymmetric about the $x$-axis and the time-averaged velocity components
    are: the velocity in the streamwise direction ($v_x$), the velocity in the
    radial direction ($v_r$), and the azimuthal velocity ($v_\theta$). If false,
    the Cartesian coordinate velocities ($v_x$, $v_y$, $v_z$) are the
    time-averaged velocity components.
    \descriptionend
    %%%%%%%%%%%%%%%%%%%%%%%%%%%%%%%%%%%%%%%%%%%%%%%%%%%%%%%%%%%%%%%%%%%%%%%%%%%%

    %%%%%%%%%%%%%%%%%%%%%%%%%%%%%%%%%%%%%%%%%%%%%%%%%%%%%%%%%%%%%%%%%%%%%%%%%%%%
    %%%%%%%%%%%%%%%%%%%%%%%%%%%%%%%%%%%%%%%%%%%%%%%%%%%%%%%%%%%%%%%%%%%%%%%%%%%%
    %%%%%%%%%%%%%%%%%%%%%%%%%%%%%%%%%%%%%%%%%%%%%%%%%%%%%%%%%%%%%%%%%%%%%%%%%%%%
    \groupline{UTILITY TO TIME AVERAGE EXISTING RESTART FILES}
    %===========================================================================
    time\_average\_restart\_files & \typlog & \fls  & \slutl &
    \descriptionbegin
    Logical flag to run a utility that creates a time-averaged solution from a
    list of existing restart files given in the file specified by the
    \textsl{restart\_list\_file} input parameter. This utility will finish by
    writing out a time-averaged restart file and a time-averaged CGNS solution
    file. This utility does not execute anything within the solver.
    \descriptionend
    %---------------------------------------------------------------------------
    restart\_list\_file           & \tc     & `` '' & \slutl &
    \descriptionbegin
    File path to a simple ASCII file that contains a list of file paths to
    existing restart files that will be used to create a time-averaged solution.
    This is only used if the input flag \textsl{time\_average\_restart\_files}
    is true.
    \descriptionend
    %%%%%%%%%%%%%%%%%%%%%%%%%%%%%%%%%%%%%%%%%%%%%%%%%%%%%%%%%%%%%%%%%%%%%%%%%%%%

    %%%%%%%%%%%%%%%%%%%%%%%%%%%%%%%%%%%%%%%%%%%%%%%%%%%%%%%%%%%%%%%%%%%%%%%%%%%%
    %%%%%%%%%%%%%%%%%%%%%%%%%%%%%%%%%%%%%%%%%%%%%%%%%%%%%%%%%%%%%%%%%%%%%%%%%%%%
    %%%%%%%%%%%%%%%%%%%%%%%%%%%%%%%%%%%%%%%%%%%%%%%%%%%%%%%%%%%%%%%%%%%%%%%%%%%%
    \groupline{MEMORY USAGE}
    %===========================================================================
    dump\_memory\_usage & \typlog & \fls & \sldev &
    \descriptionbegin
    Flag used to profile the memory usage of GFR.
    \descriptionend
    %%%%%%%%%%%%%%%%%%%%%%%%%%%%%%%%%%%%%%%%%%%%%%%%%%%%%%%%%%%%%%%%%%%%%%%%%%%%

    %%%%%%%%%%%%%%%%%%%%%%%%%%%%%%%%%%%%%%%%%%%%%%%%%%%%%%%%%%%%%%%%%%%%%%%%%%%%
    %%%%%%%%%%%%%%%%%%%%%%%%%%%%%%%%%%%%%%%%%%%%%%%%%%%%%%%%%%%%%%%%%%%%%%%%%%%%
    %%%%%%%%%%%%%%%%%%%%%%%%%%%%%%%%%%%%%%%%%%%%%%%%%%%%%%%%%%%%%%%%%%%%%%%%%%%%
    \groupline{DUMP FLUX INFORMATION FOR DEBUGGING}
    %===========================================================================
    dump\_fluxes                    & \typlog & \fls & \sldev &
    \descriptionbegin
    Flag used to output the fluxes computed in the solver.
    \descriptionend
    %---------------------------------------------------------------------------
    check\_flux\_point\_coordinates & \typlog & \fls & \sldev &
    \descriptionbegin
    Flag used to output some files to check that co-located face quantities
    are consistent in terms of interpolation between two cells sharing a face.
    \descriptionend
    %%%%%%%%%%%%%%%%%%%%%%%%%%%%%%%%%%%%%%%%%%%%%%%%%%%%%%%%%%%%%%%%%%%%%%%%%%%%

    %%%%%%%%%%%%%%%%%%%%%%%%%%%%%%%%%%%%%%%%%%%%%%%%%%%%%%%%%%%%%%%%%%%%%%%%%%%%
    %%%%%%%%%%%%%%%%%%%%%%%%%%%%%%%%%%%%%%%%%%%%%%%%%%%%%%%%%%%%%%%%%%%%%%%%%%%%
    %%%%%%%%%%%%%%%%%%%%%%%%%%%%%%%%%%%%%%%%%%%%%%%%%%%%%%%%%%%%%%%%%%%%%%%%%%%%
    \groupline{ACTIVATE SPECIAL GRID FOR DEBUGGING THE POSTPROCESSOR}
    %===========================================================================
    postproc\_debug\_grid & \typlog & \fls & \sldev &
    \descriptionbegin
    Flag used to activate using an internally created grid that is useful for
    debugging the post-processor used for the \textsl{continuous\_output} input
    flag.
    \descriptionend
    %%%%%%%%%%%%%%%%%%%%%%%%%%%%%%%%%%%%%%%%%%%%%%%%%%%%%%%%%%%%%%%%%%%%%%%%%%%%

    %%%%%%%%%%%%%%%%%%%%%%%%%%%%%%%%%%%%%%%%%%%%%%%%%%%%%%%%%%%%%%%%%%%%%%%%%%%%
    %%%%%%%%%%%%%%%%%%%%%%%%%%%%%%%%%%%%%%%%%%%%%%%%%%%%%%%%%%%%%%%%%%%%%%%%%%%%
    %%%%%%%%%%%%%%%%%%%%%%%%%%%%%%%%%%%%%%%%%%%%%%%%%%%%%%%%%%%%%%%%%%%%%%%%%%%%
    \groupline{OPTION TO OUTPUT PLOT3D GRID TO TECPLOT IN UNSTRUCTURED FORMAT}
    %===========================================================================
    output\_plot3d\_to\_tecplot & \typlog & \fls & \slutl &
    \descriptionbegin
    Flag to output a Tecplot file of the input Plot3D grid file after reading
    it.
    \descriptionend
    %%%%%%%%%%%%%%%%%%%%%%%%%%%%%%%%%%%%%%%%%%%%%%%%%%%%%%%%%%%%%%%%%%%%%%%%%%%%

    %%%%%%%%%%%%%%%%%%%%%%%%%%%%%%%%%%%%%%%%%%%%%%%%%%%%%%%%%%%%%%%%%%%%%%%%%%%%
    %%%%%%%%%%%%%%%%%%%%%%%%%%%%%%%%%%%%%%%%%%%%%%%%%%%%%%%%%%%%%%%%%%%%%%%%%%%%
    %%%%%%%%%%%%%%%%%%%%%%%%%%%%%%%%%%%%%%%%%%%%%%%%%%%%%%%%%%%%%%%%%%%%%%%%%%%%
    \groupline{OPTION TO OUTPUT BOUNDARY CONDITIONS TO TECPLOT FILE}
    %===========================================================================
    output\_bface\_array & \typlog & \fls & \sldev &
    \descriptionbegin
    Flag to have each MPI process output the contents of its own local bface
    array to the text file ``bface.[MPI process \#].dat''. The bface array
    contains the boundary conditions and connectivity information for each
    boundary face.
    \descriptionend
    %%%%%%%%%%%%%%%%%%%%%%%%%%%%%%%%%%%%%%%%%%%%%%%%%%%%%%%%%%%%%%%%%%%%%%%%%%%%

    %%%%%%%%%%%%%%%%%%%%%%%%%%%%%%%%%%%%%%%%%%%%%%%%%%%%%%%%%%%%%%%%%%%%%%%%%%%%
    %%%%%%%%%%%%%%%%%%%%%%%%%%%%%%%%%%%%%%%%%%%%%%%%%%%%%%%%%%%%%%%%%%%%%%%%%%%%
    %%%%%%%%%%%%%%%%%%%%%%%%%%%%%%%%%%%%%%%%%%%%%%%%%%%%%%%%%%%%%%%%%%%%%%%%%%%%
    \groupline{INPUT BOUNDARY CONDITIONS}
    %===========================================================================
    bc\_input(1:20)              & \typebc &       & \slbsc &
    \begin{minipage}[t]{\linewidth}\begin{flushleft}
    Derived type to specify boundary conditions.\newline
    An iterative loop will go through the specified flow conditions and try to
    compute the remaining unspecified flow conditions. \newline
    \NOTE A run-time error will occur if either of the following conditions are
    true: \newline \newline
    \begin{tabular}{ @{\qquad} r @{) } p{0.85\linewidth} @{} }
    A & The flow conditions are over-specified and inconsistent \\
    B & The flow conditions are under-specified and it is unable to compute the
    remaining unspecified flow conditions.
    \end{tabular} \newline \newline
    \EXAMPLE An example of a well defined boundary condition using bc\_input:
    \newline \newline
    \begin{tabular}{ @{\qquad} l @{ = } l @{} }
    bc\_input(1)\%name & ``INFLOWSUBSONIC'' \\
    bc\_input(1)\%set\_bc\_value & 2 ! Subsonic Inflow BC \\
    bc\_input(1)\%p\_static & 101325.0 \\
    bc\_input(1)\%p\_total & 121286.025 \\
    bc\_input(1)\%t\_total & 293.15
    \end{tabular}
    \end{flushleft}\end{minipage} \\ \minorline
    %---------------------------------------------------------------------------
    %%%%%%%%%%%%%%%%%%%%%%%%%%%%%%%%%%%%%%%%%%%%%%%%%%%%%%%%%%%%%%%%%%%%%%%%%%%%
    bc\_input(1)\%name           & \tc[32] & `` '' & \slbsc &
    \descriptionbegin
    Character string that matches the name of a boundary group in the grid file.
    \NOTE Case (upper\slash lower-case) does NOT matter.
    \descriptionend
    %%%%%%%%%%%%%%%%%%%%%%%%%%%%%%%%%%%%%%%%%%%%%%%%%%%%%%%%%%%%%%%%%%%%%%%%%%%%
    bc\_input(1)\%set\_bc\_value & \typint & -9999 & \slbsc &
    \begin{minipage}[t]{\linewidth}\begin{flushleft}
    This overrides the boundary condition type specified in the grid file, using
    this boundary condition type for the boundary group given by
    \textsl{bc\_input(1)\%name}. Valid values for boundary condition types are:
    \begin{tabular}{ @{\qquad} r @{ = } p{0.845\linewidth} @{} }
    1 & Generic inflow, subsonic or supersonic  \\
    2 & Subsonic inflow \\
    3 & Supersonic inflow \\
    11 & Generic outflow, subsonic or supersonic \\
    12 & subsonic outflow \\
    13 & supersonic outflow \\
    21 & Characteristic inflow\slash outflow based on
         \newline incoming\slash outgoing Riemann invariants,
         \newline can be subsonic or supersonic \\
    22 & Generic inflow\slash outflow, subsonic or supersonic \\
    23 & Fixed to freestream\slash reference flow conditions \\
    24 & Fixed flow conditions (equivalent to supersonic inflow) \\
    50 & Slip wall \\
    52 & Adiabatic no-slip wall \\
    53 & Isothermal no-slip wall \\
    80 & Symmetry boundary condition (equivalent to slip wall)
    \end{tabular}
    \end{flushleft}\end{minipage} \\ \minorline
    %%%%%%%%%%%%%%%%%%%%%%%%%%%%%%%%%%%%%%%%%%%%%%%%%%%%%%%%%%%%%%%%%%%%%%%%%%%%
    bc\_input(1)\%t\_static      & \typflt & -1.0  & \slbsc &
    \descriptionbegin
    Static temperature (units of Kelvin)
    \descriptionend
    %---------------------------------------------------------------------------
    bc\_input(1)\%p\_static      & \typflt & -1.0  & \slbsc &
    \descriptionbegin
    Static pressure (units of Pa)
    \descriptionend
    %---------------------------------------------------------------------------
    bc\_input(1)\%rho\_static    & \typflt & -1.0  & \slbsc &
    \descriptionbegin
    Static density (units of $\frac{\textrm{kg}}{\textrm{m}^3}$)
    \descriptionend
    %---------------------------------------------------------------------------
    bc\_input(1)\%mach           & \typflt & -1.0  & \slbsc &
    \descriptionbegin
    Mach number (dimensionless)
    \descriptionend
    %---------------------------------------------------------------------------
    bc\_input(1)\%vx             & \typflt & 0.0   & \slbsc &
    \descriptionbegin
    Velocity in the $x$ coordinate direction (units of m/s)
    \descriptionend
    %---------------------------------------------------------------------------
    bc\_input(1)\%vy             & \typflt & 0.0   & \slbsc &
    \descriptionbegin
    Velocity in the $y$ coordinate direction (units of m/s)
    \descriptionend
    %---------------------------------------------------------------------------
    bc\_input(1)\%vz             & \typflt & 0.0   & \slbsc &
    \descriptionbegin
    Velocity in the $z$ coordinate direction (units of m/s)
    \descriptionend
    %---------------------------------------------------------------------------
    bc\_input(1)\%t\_total       & \typflt & -1.0  & \slbsc &
    \descriptionbegin
    Total temperature (units of Kelvin)
    \descriptionend
    %---------------------------------------------------------------------------
    bc\_input(1)\%p\_total       & \typflt & -1.0  & \slbsc &
    \descriptionbegin
    Total pressure (units of Pa)
    \descriptionend
    %---------------------------------------------------------------------------
    bc\_input(1)\%rho\_total     & \typflt & -1.0  & \slbsc &
    \descriptionbegin
    Total density (units of $\frac{\textrm{kg}}{\textrm{m}^3}$)
    \descriptionend
    %---------------------------------------------------------------------------
    bc\_input(1)\%alpha\_aoa     & \typflt & 0.0   & \slbsc &
    \descriptionbegin
    Angle of attack with respect to the $x$-$y$ plane (dimensionless)
    \descriptionend
    %---------------------------------------------------------------------------
    bc\_input(1)\%beta\_aoa      & \typflt & 0.0   & \slbsc &
    \descriptionbegin
    Angle of attack with respect to the $x$-$z$ plane (dimensionless)
    \descriptionend
    %---------------------------------------------------------------------------
    bc\_input(1)\%wall\_temp     & \typflt & -1.0  & \slbsc &
    \descriptionbegin
    Temperature for isothermal wall (units of Kelvin)
    \descriptionend
    %%%%%%%%%%%%%%%%%%%%%%%%%%%%%%%%%%%%%%%%%%%%%%%%%%%%%%%%%%%%%%%%%%%%%%%%%%%%
    bc\_input(1)\%relax\%value   & \typflt & 1.0   & \slxtl &
    \descriptionbegin
    Relaxation value to limit the difference between the interior and exterior
    states ($0.0 < \textrm{bc\_input(1)\%relax\%value} \leq 1.0$)
    \descriptionend
    %---------------------------------------------------------------------------
    bc\_input(1)\%relax\%time    & \typflt & 0.0   & \slxtl &
    \descriptionbegin
    Solution time to end relaxation ($> 0.0$)
    \descriptionend
    %---------------------------------------------------------------------------
    bc\_input(1)\%relax\%iter    & \typint & 0     & \slxtl &
    \descriptionbegin
    Time step to end relaxation ($> 0$)
    \descriptionend
    %%%%%%%%%%%%%%%%%%%%%%%%%%%%%%%%%%%%%%%%%%%%%%%%%%%%%%%%%%%%%%%%%%%%%%%%%%%%



   %Monday & 11C & 22C & A clear day with lots of sunshine.  
   %However, the strong breeze will bring down the temperatures. \\ \hline
   %Tuesday & 9C & 19C & Cloudy with rain, across many northern regions.
   %Clear spells across most of Scotland and Northern Ireland, 
   %but rain reaching the far northwest. \\ \hline
   %Wednesday & 10C & 21C & Rain will still linger for the morning. 
   %Conditions will improve by early afternoon and continue 
   %throughout the evening. \\
    \hline
%   \end{tabularx}
    \end{longtable}
%   \end{tabular}
%\end{center}

%\input{vortex}
%
%\newpage
%
%\input{taylorgreen}

\end{document}
